\documentclass[11pt,answers]{exam}


%\usepackage[margin=0.4in,footskip=0.2in,]{geometry}
\usepackage[a4paper,bindingoffset=0.2in,%
            left=0.2in,right=0.2in,top=0.6in,bottom=0.2in,%
            footskip=.2in]{geometry}

\usepackage{etex}
\usepackage{amssymb,amsmath,multicol} %<-- InWorksheetExam1 i also have fancyhdr,

\usepackage[metapost]{mfpic}
\usepackage[pdftex]{graphicx}

\usepackage{pst-plot}
\usepackage{pgfplots}
\pgfplotsset{compat=1.9}

\usepackage{tikz}
\usepackage{tkz-2d}
\usepackage{tkz-base}
\usetikzlibrary{calc}

\usepackage{tabularx, booktabs}
\newcolumntype{Y}{>{\centering\arraybackslash}X}

\usepackage[inline]{enumitem}
\usepackage{refcount}%<-- non in WorksheetExam1

\usepackage{pstricks-add,pst-eucl}
\usepackage{caption}


\usepackage{setspace}
\renewcommand{\baselinestretch}{1.50}\normalsize

\def\f{x+1} \def\g{-x/3+2}  \def\h{-x+3}

\newcommand{\vasymptote}[2][]{
	\draw [densely dashed,#1] ({rel axis cs:0,0} -| {axis cs:#2,0}) -- ({rel axis cs:0,1} -| {axis cs:#2,0});
}

\newcommand*\circled[1]{\tikz[baseline=(char.base)]{%
		\node[shape=circle,draw,inner sep=1pt] (char) {#1};}}
\renewcommand\choicelabel{\circled{\thechoice}}


\let\originalleft\left
\let\originalright\right
\renewcommand{\left}{\mathopen{}\mathclose\bgroup\originalleft}
\renewcommand{\right}{\aftergroup\egroup\originalright}

\addpoints
%\printanswers
\noprintanswers

%%\usepackage{geometry}
%%\geometry{
	%a4paper,
	%total={170mm,257mm},
	%left=20mm,
	%top=20mm,
%%	text={.95\paperwidth,.95\paperheight}, ratio=1:3,%includefoot
%%}

\opengraphsfile{Q9a_Spring_16}

\begin{document}
%\extrawidth{-0.3in}
\pagestyle{headandfoot}

%\setlength{\hoffset}{-.25in}

%\extraheadheight{-.3in}
\runningheadrule
\firstpageheader{\bfseries {MATH1-UC 1171}}{ \bfseries {Quiz 9 }}{\bfseries {4/12/2016}} 



\firstpagefooter{} %%&&CHANGED
                {}
                {Points earned: \hbox to 0.5in{\hrulefill}
                 out of  \pointsonpage{\thepage} points}
                 
						

%\vspace*{0.1cm}
\hbox to \textwidth { \scshape {Name:} \enspace\hrulefill}
\vspace{0.1in}




\pointpoints{point}{points}

%\def\baselinestretch{1}\selectfont
%\doublespacing
\begin{questions}


\addpoints

\question[2] The equation 
$\displaystyle N(t) = \frac{700}{1 + 69e^{−0.6t}}$
models the number of people $N$ in a town who have heard a rumor after $t$ days. 

How many people started the rumor? Show your work step by step.
\fillwithdottedlines{2cm}
\question If an initial principal $P$ is invested at an annual rate r and the interest is compounded $n$ times per year, the amount $A$ 
in the account $t$ years after May 31, 2009 is
$\displaystyle A(t) = P\left(1 + \frac{r}{n}\right )^{nt}$.
On May 31, 2009, the Annual Percentage Rate listed at Jeff's bank for regular savings accounts was 0.24\% compounded monthly, and Jeff invests \$4000.
\begin{parts}
	\part[1] Write the formula for $A(t)$. \dotfill
	\part[2] \label{part:odifreddi} Solve the equation $A(t)=6000$ for $t$. (Your answer will be a fraction.)
	\fillwithdottedlines{3cm}
	\bonuspart[1] What does your answer in part~(\ref{part:odifreddi}) represent in practical terms?
	\fillwithdottedlines{1cm}
	\end{parts}

\question The population of Burkina Faso is about 18.9 million and grows at a rate of 3.03\% per year. The population of Canada is approximately 35 millions, and grows at a rate of 0.75\% per year. (The size of Burkina Faso is 105,869 square  miles, and the size of Canada is 3,854,085 square miles.) 
\begin{parts}
	\part[2] \label{part:perfetto} Write an equation that allows you to find when the population of Burkina Faso and the population of Canada are equal. \dotfill
	\bonuspart[2] Solve the equation in part (\ref{part:perfetto}).
	\fillwithdottedlines{2cm}
\end{parts}
%%%%%%%%%%%%%%%%%%%%%%
\question[2] Solve the equation $\displaystyle \log(x)+\log(10x)=10$.
\fillwithdottedlines{3cm}

\end{questions}

\end{document}                 