\documentclass[11pt,dvipsnames]{article}

\usepackage{etex}
\usepackage{amssymb,amsmath,multicol} %<-- InWorksheetExam1 i also have fancyhdr,

\usepackage[metapost]{mfpic}
\usepackage[pdftex]{graphicx}

\usepackage{pst-plot}
\usepackage{pgfplots}
\pgfplotsset{compat=1.9}

\usepackage{tikz}
\usepackage{tkz-2d}
\usepackage{tkz-base}
\usetikzlibrary{calc}
\usepackage{color}
\usepackage[inline]{enumitem}
\usepackage{refcount}%<-- non in WorksheetExam1

\usepackage{caption}
%\renewcommand{\headrulewidth}{0pt}

%%These three lines are for the typewriter font. Comment them out if I don't want the font.
%%%%%%\renewcommand*\ttdefault{lcmtt}
%%%%%%\renewcommand*\familydefault{\ttdefault} %% Only if the base font of the document is to be typewriter style
%%%%%%\usepackage[T1]{fontenc}
%%%%%%

\newcommand{\vasymptote}[2][]{
    \draw [densely dashed,#1] ({rel axis cs:0,0} -| {axis cs:#2,0}) -- ({rel axis cs:0,1} -| {axis cs:#2,0});
}

\newcommand{\inlineitem}[1][]{%
\ifnum\enit@type=\tw@
    {\descriptionlabel{#1}}
  \hspace{\labelsep}%
\else
  \ifnum\enit@type=\z@
       \refstepcounter{\@listctr}\fi
    \quad\@itemlabel\hspace{\labelsep}%
\fi}

\usetikzlibrary{automata}

\usepackage{hyperref}% http://ctan.org/pkg/hyperref

\usepackage{geometry}
 \geometry{
 	%a4paper,
 	%total={170mm,257mm},
 	%left=20mm,
 	%top=20mm,
text={.8\paperwidth,.9\paperheight}, ratio=1:1,includefoot
 }

\opengraphsfile{Wk_2B_Questions_Spring_16}

\begin{document}
%\thispagestyle{empty}
Your book gives this mathematical definition of the word function: a function is a rule that assigns to each element of a set A (let's say a generic element is called $x$) a unique element (which we call $f(x)$, and read "f of x") of a set B (the set B may coincide with A, but it doesn't have to).

The two key words here are {\textcolor{blue}{\underline{each} }} and {\textcolor{blue}{\underline{unique}}}.

For example, the temperature in degrees Fahrenheit is a function in of the temperature in degrees Celsius, because each temperature in 
Celsius corresponds to one temperature in Fahrenheit (the converse is true as well, that is, the temperature in degrees Celsius is a 
function of the temperature in degrees Fahrenheit).

\begin{enumerate}[label=$\blacktriangleright$ {\bf  \arabic*:}]
	\item Is the amount of money in a bank account a function of the account number? Is the account number a function of the amount of money in the account?

\item In this problem, the four graphs don't have tick marks and units on the vertical axis. Solve the problem and complete the graphs by adding appropriate tick marks and units on the vertical axis.



\begin{figure}[ht] 
	\label{ fig7} 
	\begin{minipage}[b]{0.5\linewidth}
		\centering
		\includegraphics[width=.5\linewidth]{Pic1.png} 
		\caption{} 
		\vspace{4ex}
	\end{minipage}%%
	\begin{minipage}[b]{0.5\linewidth}
		\centering
		\includegraphics[width=.5\linewidth]{Pic2.png} 
		\caption{} 
		\vspace{4ex}
	\end{minipage} 
	\begin{minipage}[b]{0.5\linewidth}
		\centering
		\includegraphics[width=.5\linewidth]{Pic3.png} 
		\caption{} 
		\vspace{4ex}
	\end{minipage}%% 
	\begin{minipage}[b]{0.5\linewidth}
		\centering
		\includegraphics[width=.5\linewidth]{Pic4.png} 
		\caption{} 
		\vspace{4ex}
	\end{minipage} 
\end{figure}



\begin{enumerate}[label=$\blacktriangleright$ {\bf  \Alph*:}] 
	\item Match each of the stories listed below with one of the graphs shown above.
	\begin{enumerate}[label= \fbox{\bf Story \arabic*}]
		\item 	I took some frozen vegetables out of the freezer at noon and left them on the counter to thaw. Then I cooked them in the oven when I got home.
		\item 	I took some frozen vegetables out of the freezer in the morning and left them on the counter to thaw. Then I cooked them in the oven when I got home.
		\item I took some frozen vegetables out of the freezer in the morning and left them on the counter to thaw. I forgot about them and went out for pizza on my way home from work. I put the vegetables in the refrigerator when I finally got home.
\end{enumerate}	
	\item Write a story for the unmatched graph.
\end{enumerate}	

\item One of the difficulties we face when we model something using math formulas is that we don't have complete information, and so we have to fill it in by making reasonable working assumptions. Here are two examples.




\begin{enumerate}[label=$\blacktriangleright$ {\bf  \Alph*:}] 
	\item Log on to \url{student.desmos.com} and enter the code: dgc7
	\smallskip Complete the activities on the website by Monday Feb 1, 11:59 pm. Record your answers and be ready to share them in class on Tuesday.
	\item I want to fill this vase \url{http://amzn.to/1P69Szp} with water. Draw a graph representing the height of the water in the vase as time goes by. On the horizontal axis you should record the time, while the height of the water is on the vertical axis. Make sure you include units and list the assumptions you are making in order to come up with the graph.
\end{enumerate}
\end{enumerate}	
\end{document} 
              