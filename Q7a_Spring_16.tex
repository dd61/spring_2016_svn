\documentclass[11pt,answers]{exam}


\usepackage[margin=0.8in,footskip=0.2in]{geometry}

\usepackage{etex}
\usepackage{amssymb,amsmath,multicol} %<-- InWorksheetExam1 i also have fancyhdr,

\usepackage[metapost]{mfpic}
\usepackage[pdftex]{graphicx}

\usepackage{pst-plot}
\usepackage{pgfplots}
\pgfplotsset{compat=1.9}

\usepackage{tikz}
\usepackage{tkz-2d}
\usepackage{tkz-base}
\usetikzlibrary{calc}

\usepackage{tabularx, booktabs}
\newcolumntype{Y}{>{\centering\arraybackslash}X}

\usepackage[inline]{enumitem}
\usepackage{refcount}%<-- non in WorksheetExam1

\usepackage{pstricks-add,pst-eucl}
\usepackage{caption}

\def\f{x+1} \def\g{-x/3+2}  \def\h{-x+3}

\newcommand{\vasymptote}[2][]{
	\draw [densely dashed,#1] ({rel axis cs:0,0} -| {axis cs:#2,0}) -- ({rel axis cs:0,1} -| {axis cs:#2,0});
}

\newcommand*\circled[1]{\tikz[baseline=(char.base)]{%
		\node[shape=circle,draw,inner sep=1pt] (char) {#1};}}
\renewcommand\choicelabel{\circled{\thechoice}}


\let\originalleft\left
\let\originalright\right
\renewcommand{\left}{\mathopen{}\mathclose\bgroup\originalleft}
\renewcommand{\right}{\aftergroup\egroup\originalright}

\addpoints
%\printanswers
\noprintanswers

\usepackage{geometry}
\geometry{
	%a4paper,
	%total={170mm,257mm},
	%left=20mm,
	%top=20mm,
	text={.95\paperwidth,.95\paperheight}, ratio=1:3,%includefoot
}

\opengraphsfile{Q7a_Spring_16}

\begin{document}
%\extrawidth{-0.3in}
\pagestyle{headandfoot}

%\setlength{\hoffset}{-.25in}

%\extraheadheight{-.3in}
\runningheadrule
\firstpageheader{\bfseries {MATH1-UC 1171}}{ \bfseries {Quiz 7 }}{\bfseries {3/29/2016}} 



\firstpagefooter{} %%&&CHANGED
                {}
                {Points earned: \hbox to 0.5in{\hrulefill}
                 out of  \pointsonpage{\thepage} points}
                 
						

%\vspace*{0.1cm}
\hbox to \textwidth { \scshape {Name:} \enspace\hrulefill}
\vspace{0.1in}




\pointpoints{point}{points}

\begin{questions}


\addpoints

\question[2] If a client of a personal trainer signs up for more than one session, then the charge for each session is \$2 less than the previous training session (the first session costs \$100). A client signs up for 17 sessions. Write a formula that gives the cost of the seventeenth session. (You don't need to simplify the formula.)
\fillwithdottedlines{2cm}
\question[2] If a client of a personal trainer signs up for more than one session, then the charge for each session is 2\% less than the previous training session (the first session costs \$100). A client signs up for 17 sessions. Write a formula that gives the cost of the seventeenth session. (You don't need to simplify the formula.)
\fillwithdottedlines{2cm}
\question[1] The equation $\displaystyle y=200(1-x)^5$ represents:
\begin{oneparchoices}
\choice exponential growth; \choice linear growth; \choice neither.
\end{oneparchoices}
\question Suppose a single bacterium is placed in a bottle at 11:00 am. It grows
and at 11:01 am it divides into two bacteria. The two bacteria grow
and at 11:02 am each of them divides into two bacteria, and so on. The
bacteria continues to double every minute, and the bottle is full at noon.
\begin{parts}
\part[1] At what time is the bottle half full? \dotfill
\part[1] It is 11:56 am. What fraction of the bottle is full at this time? \dotfill
\end{parts}
\question The function $g(x)$ is defined by the formula: $\displaystyle g(x) = -2^{x-1} + 2$. 
\begin{parts}
\bonuspart[1] Write the domain of $g(x)$ in interval form. \dotfill
\part[2] What transformations must be applied to $f(x)=2^x$ to get $g(x)$?
\fillwithdottedlines{2cm}
\bonuspart[1] Write the range of $g(x)$ in interval form. \dotfill

\end{parts}
%%%%%%%%%%%%%%%%%%%%%%

\end{questions}

\end{document}                 