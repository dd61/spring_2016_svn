\documentclass[11pt,dvipsnames]{article}

\usepackage{etex}
\usepackage{amssymb,amsmath,multicol} %<-- InWorksheetExam1 i also have fancyhdr,

\usepackage[metapost]{mfpic}
\usepackage[pdftex]{graphicx}

\usepackage{pst-plot}
\usepackage{pgfplots}
\pgfplotsset{compat=1.9}

\usepackage{tikz}
\usepackage{tkz-2d}
\usepackage{tkz-base}
\usetikzlibrary{calc}
\usepackage{color}
\usepackage[inline]{enumitem}
\usepackage{refcount}%<-- non in WorksheetExam1

\usepackage[linewidth=1pt]{mdframed}

\usepackage{caption}
\usepackage{csquotes}

\usepackage{frcursive}
\usepackage[T1]{fontenc}
%\newcommand{\setfont}[2]{{\fontfamily{#1}\selectfont #2}}
\newenvironment{myfont}{\fontfamily{frc}\selectfont}{\par}
%\renewcommand{\headrulewidth}{0pt}

%%These three lines are for the typewriter font. Comment them out if I don't want the font.
%%%%%%\renewcommand*\ttdefault{lcmtt}
%%%%%%\renewcommand*\familydefault{\ttdefault} %% Only if the base font of the document is to be typewriter style
%%%%%%\usepackage[T1]{fontenc}
%%%%%%

\usepackage{tabularx, booktabs}
\newcolumntype{Y}{>{\centering\arraybackslash}X}

\newcommand{\vasymptote}[2][]{
	\draw [densely dashed,#1] ({rel axis cs:0,0} -| {axis cs:#2,0}) -- ({rel axis cs:0,1} -| {axis cs:#2,0});
}

\newcommand{\inlineitem}[1][]{%
	\ifnum\enit@type=\tw@
	{\descriptionlabel{#1}}
	\hspace{\labelsep}%
	\else
	\ifnum\enit@type=\z@
	\refstepcounter{\@listctr}\fi
	\quad\@itemlabel\hspace{\labelsep}%
	\fi}

\usetikzlibrary{automata}

\usepackage{hyperref}% http://ctan.org/pkg/hyperref

\usepackage{geometry}
\geometry{
	%a4paper,
	%total={170mm,257mm},
	%left=20mm,
	%top=20mm,
	text={.8\paperwidth,.9\paperheight}, ratio=1:1,includefoot
}


\opengraphsfile{Wk5_Overview_Spring_16}

\begin{document}
%\thispagestyle{empty}
\begin{center}
\begin{myfont}
{\Large Week 5 Overview }	
\end{myfont}
\end{center}


For the rest of the semester, we will study specific types of functions: polynomial functions, rational functions, exponential and logarithmic functions, and trigonometric functions. The simplest examples of polynomial functions are lines ($y = mx + b$) and the parabola  $\displaystyle y = x^2$. We create polynomials by adding, subtracting and multiplying \enquote{blocks} (called monomials) made of numbers multiplied by positive integer powers of x. For example, $\displaystyle y = 3x^4 - \pi x^5$ is a polynomial, but $\displaystyle y=\frac{3}{x}$ is not. The highest power of $x$ is called the degree.


Working with polynomials of degree two (parabolas) is much easier that working with polynomials of higher degrees. For example, we can graph a parabola precisely, finding the $x$ intercepts, the y intercept, the lowest (or highest) point, and the range. However, if we have a polynomial of degree greater than two, it may happen that we know there must be $x$-intercepts, but we are not able to find them; we may also not be able to determine a precise interval for the range, or sketch an accurate graph. We will spend a lot of time trying to discover features of a polynomial of degree greater than 3, and sketch graphs that are qualitatively accurate, even though they are not precise. For example: does $\displaystyle y = 3x^4 - \pi x^5$ have any $x$ intercept other that $x = 0$? If so, how many? How does the graph look like if we choose larger and larger values for $x$? Does the graph have any peaks or valleys? 

Polynomials are widely used in economics (I posted a paper on polynomial cost functions) and in computer science (here is the Wikipedia link for {\emph{time complexity}} \url{https://en.wikipedia.org/wiki/Time_complexity#Polynomial_time}.) 

We can also approximate trends of data sets using polynomials: this shows us trends in the data and allows us to make predictions.


\end{document}                 