\documentclass[10pt,answers]{exam}
%\usepackage{mystyleTest}
\usepackage{etex}
\usepackage{amssymb,amsmath,multicol} %<-- InWorksheetExam1 i also have fancyhdr,

\usepackage{hyperref}


\usepackage[metapost]{mfpic}
\usepackage[pdftex]{graphicx}

\usepackage{pst-plot}
\usepackage{pgfplots}
\pgfplotsset{compat=1.9}

\usepackage{tikz}
\usepackage{tkz-2d}
\usepackage{tkz-base}
\usetikzlibrary{calc}

\usepackage[inline]{enumitem}
\usepackage{refcount}%<-- non in WorksheetExam1
\usepackage{subcaption}

\usepackage{tabularx, booktabs}



\newcommand*\circled[1]{\tikz[baseline=(char.base)]{%
		\node[shape=circle,draw,inner sep=1pt] (char) {#1};}}
\renewcommand\choicelabel{\circled{\thechoice}}

\CorrectChoiceEmphasis{\color{red}}


\setlength\answerlinelength{2in}
\addpoints
%\printanswers
\noprintanswers

\begin{document}

\extrawidth{-0.3in}
\pagestyle{headandfoot}

\setlength{\hoffset}{-.25in}

\extraheadheight{-.4in}

%%\pagestyle{headandfoot}
\extraheadheight{-0.9cm}
%%\runningheadrule
%%\firstpageheader{\bfseries {}}{ \bfseries {NYU SPS}}{\bfseries {McGhee Division}} 

\parbox[b]{1.75in}{%
	\vspace{0.5in}%
	\includegraphics[scale=0.7,ext=.png]
	{nyu-scps-logo-lg.png}%
}%


\begin{center}
	{\bfseries {Practice Problems \#2}}
\end{center}
\bigskip
 

%\vspace*{0.2cm}
%\hbox to \textwidth { \scshape {Name:} \enspace\hrulefill}
%\vspace{0.7cm}


\begin{questions}

\question 
A new car was sold at \$30,000. Assume that the value of the car depreciates by 5\% per year. What is the value of the car two years after the purchase?

\begin{oneparchoices}
	\choice \$28,500
	\choice \$28,000
	\CorrectChoice \$27,075
	\choice \$27,000
	\choice \$26,500
	\end{oneparchoices}
	
\question True or false? $\displaystyle (-5)^2=-5^2$.
\begin{oneparchoices}
\choice True
\CorrectChoice False
\end{oneparchoices}

\question The formula $\displaystyle C=12N+5$ represents the total cost (in \$) 	of meals for $N$ people. 
\begin{parts}
	\part The unit of measure for 12 in the formula is:
	
	\begin{choices}
		\choice Dollars;
		\choice Number of people;
		\CorrectChoice Dollars per person;
		\choice Number of meals per dollar.
		\end{choices}
\part The unit of measure for 5 in the formula is:
	\begin{choices}
		\CorrectChoice Dollars;
		\choice Number of people;
		\choice Dollars per person;
		\choice Number of meals per dollar.
	\end{choices}	
	\part According to the formula, how many meals can be purchased for \$113?
	
	\begin{oneparchoices}
		\choice 11
		\choice 10
		\CorrectChoice 9
		\choice 8
		\end{oneparchoices}
	
	\end{parts}
	
	\question If I run two and a half kilometers per day, in how many days will I run 55 kilometers?
	
	\begin{oneparchoices}
	\choice 	$\displaystyle 55\times 2.5$
	\choice  $\displaystyle 2.5 \div 55$
	\CorrectChoice $\displaystyle 55 \div 2.5$
	\choice None of these
		\end{oneparchoices}
	
	\question A rectangle has an area of a half square meter and one of the sides is 20 centimeters long. What is the lenght of the other side?
	
	\begin{oneparchoices}
		\choice 10 meters
		\choice 10 centimeters
		\CorrectChoice 2.5 meters
		\choice 2 meters
		
		\end{oneparchoices}

\question In 1997, The New York Times reported that the percentage of tenth graders who smoked regularly was up 45\% from 1991, to 18.3\%. This means: 


\begin{choices}
\choice The percentage of tenth graders who smoked regularly went from 45\% in 1991 to 18.3\% in 1997.
\CorrectChoice In 1991, approximately 12.62\% of tenth graders smoked regularly.
\choice The report must be incorrect, because it would mean that in 1991 the percent of tenth graders who were smokers was negative (18.3\%-45\%).
\choice The statement does not make sense, because the number of tenth graders in 1991 may be different from the number of tenth graders in 1997.
\end{choices}






\end{questions}
\end{document}