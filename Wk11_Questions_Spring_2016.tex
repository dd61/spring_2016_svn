\documentclass[12pt,dvipsnames]{article}
\usepackage[margin=0.4in,footskip=0.2in]{geometry}
\usepackage{etex}
\usepackage{amssymb,amsmath,multicol} %<-- InWorksheetExam1 i also have fancyhdr,
\usepackage{hyperref}
\usepackage[metapost]{mfpic}
\usepackage[pdftex]{graphicx}

\usepackage{pst-plot}
\usepackage{pgfplots}
\pgfplotsset{compat=1.9}

\usepackage{tikz}
\usepackage{tkz-2d}
\usepackage{tkz-base}
\usetikzlibrary{calc}
\usepackage{color}
\usepackage[inline]{enumitem}
\usepackage{refcount}%<-- non in WorksheetExam1

\usepackage[linewidth=1pt]{mdframed}

\usepackage{caption}
\usetikzlibrary{calc,fit,intersections,shapes,calc}




%%These three lines are for the typewriter font. Comment them out if I don't want the font.
%%%%%%\renewcommand*\ttdefault{lcmtt}
%%%%%%\renewcommand*\familydefault{\ttdefault} %% Only if the base font of the document is to be typewriter style
%%%%%%\usepackage[T1]{fontenc}
%%%%%%

\usepackage{tabularx, booktabs}


\newenvironment{myitemize}
{ \begin{itemize}
		\setlength{\itemsep}{10pt}
		\setlength{\parskip}{10pt}
		\setlength{\parsep}{10pt}     }
	{ \end{itemize}   
	
           } 

\usepackage{setspace}

\renewcommand{\baselinestretch}{1.50}\normalsize

\opengraphsfile{Wk11_Questions_Spring_2016}

\begin{document}

%\includegraphics*[100,100][300,300]{nyu-scps-logo-lg.png}

\begin{enumerate}[label= {\bf  \arabic*:}]
\item Last week we realized that it is close to impossible to visualize numbers of very different sizes on a linear scale. Our goal was to get a sense of the relative size of \$1,000,000 and \$1,000,000,000, so we started by drawing a number line where the tick mark 1 corresponded to one million dollars. In this line, the 1 billion mark would be the 1000 tick mark, which is very unpractical to draw. We then took the opposite approach, starting with a line where the tick mark 1 represents 1 billion (we made sure to draw the tick marks at 0.1, 0.2 etc.). Here, 1 million would be placed very close to the tick mark representing zero, and tick marks representing smaller quantities (like 10,000 and 1,000) would be almost indistinguishable from the tick mark representing zero. Then we created a new number line, where the tick mark 1 represented the exponent of $\displaystyle 10^1$, the tick mark 2 represented the exponent of $\displaystyle 10^2$, and so on. In this number line, we had no difficulty visualizing both 1,000,000 (the tick mark 6) and 1,000,000 (the tick mark 9.) We noted that this is a \emph{logarithmic} scale, because the exponents we plotted were logarithms ($\displaystyle 1=\log(10)$, $2=\log\left( 10^2\right)$, etc.)

\begin{enumerate}
	\item  Another way to label the logarithmic scale is with the values themselves instead of the exponents. 
	
	\begin{tikzpicture}
	\draw (0.5,0) -- (11.5,0);
	
	\draw[xshift=2 cm] (0pt,2pt) -- (0pt,-2pt) node[below,fill=white] {0.1};
	\draw[xshift=4 cm] (0pt,2pt) -- (0pt,-2pt) node[below,fill=white] {1};
	\draw[xshift=6 cm] (0pt,2pt) -- (0pt,-2pt) node[below,fill=white] {10};
	\draw[xshift=8 cm] (0pt,2pt) -- (0pt,-2pt) node[below,fill=white] {100};
	\draw[xshift=10 cm] (0pt,2pt) -- (0pt,-2pt) node[below,fill=white] {1000};
	
	%\node at (6.5,-1.0) {Log scale};
	\end{tikzpicture}
	
	In this number line, each tick mark is 10 times as large as the tick mark immediately preceding it. 
	
	Plot the numbers 2, 5, 8, 9 on this number line. [Hint: plot the logarithm of each number.] What do you notice about the position of the numbers you plotted? 
\item Plot the function $\displaystyle y=10^x$ on the graph paper provided below.

\begin{tikzpicture}[scale=1.7]
\begin{axis}[
 ymode=log,
 axis y line*=middle,
%xmajorticks=false,
xmin=-10, xmax=10,
ymin=1e-1, ymax=1e4,
grid=both,
major grid style={black!50},
xticklabels={-5,-4,-3,-2, -1,0,1,2,3, 4,5},xtick={-10,-8,...,10},
x tick label style={rotate=90,anchor=east}]
]
%\draw [step=1.0,blue, very thick] (0.5,0.5) grid (xmax,ymax);
\end{axis}
\end{tikzpicture}

\item Plot the function $\displaystyle y=2^x$ on the graph paper provided on the next page.

\begin{tikzpicture}[scale=1.7]
\begin{axis}[
ymode=log,
axis y line*=middle,
%xmajorticks=false,
xmin=-10, xmax=10,
ymin=1e-1, ymax=1e4,
grid=both,
major grid style={black!50},
xticklabels={-5,-4,-3,-2, -1,0,1,2,3, 4,5},xtick={-10,-8,...,10},
x tick label style={rotate=90,anchor=east}]
]
%\draw [step=1.0,blue, very thick] (0.5,0.5) grid (xmax,ymax);
\end{axis}
\end{tikzpicture}
\item The Dow Jones 100 years historical chart is available at \url{http://bit.ly/1LdgBFgf}. Deselect \emph{Inflation-Adjusted} and click on \emph{log} or \emph{linear} to view the data plotted using either a linear scale or a logarithmic scale on the vertical axis. 
\begin{enumerate}
	\item What type of function (linear, polynomial, exponential, logarithmic) best represents the graph if a linear scale is used on the vertical axis?
	\item There were stock market drops in 1929 and 2008. Which was larger? How do you know?

\end{enumerate}
\end{enumerate}
\item Between November 2014 and November 2015, the ruble lost 34\% of its value against the dollar. Suppose  the exchange rate keeps falling at the same rate. How long after November 2014 will the ruble lose half its value?
\item The monthly rate of increase in the price of ice cream is reported to be 0.2\%. Assume that the price continues growing in the same way for one year. What is the annual rate of increase? Is it 12 times the monthly rate?
\item At a roulette table in Las Vegas, you may bet on whether the number comes up even or odd. If you win, the house gives you back your bet plus a 55\% return. If you started with \$10, and reinvest all your money on each subsequent bet, how many consecutive odd-even wins does it take to become a millionaire?
\item A patient takes a medicine that has a half-life of 2 hours (that is, it takes 2 hours for concentration of the medicine in the patient's system to drop by 50\%.) An overdose occurs when the patient has a  concentration of 190mg/l (milligrams per liters of blood) of medicine in her system. Every time the patient takes a pill containing this medicine, the concentration of the medicine in the patient's system increases by 100mg/l.

At 1 pm, the patient has just taken the first dosage of the medicine, so she has a concentration of 100mg/l of medicine in her system. She takes the same amount of medicine every two hours. When will she overdose?	

\end{enumerate}

\end{document}                 