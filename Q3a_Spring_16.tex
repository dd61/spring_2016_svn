\documentclass[11pt,answers]{exam}


\usepackage[margin=0.8in,footskip=0.2in]{geometry}

\usepackage{etex}
\usepackage{amssymb,amsmath,multicol} %<-- InWorksheetExam1 i also have fancyhdr,

\usepackage[metapost]{mfpic}
\usepackage[pdftex]{graphicx}

\usepackage{pst-plot}
\usepackage{pgfplots}
\pgfplotsset{compat=1.9}

\usepackage{tikz}
\usepackage{tkz-2d}
\usepackage{tkz-base}
\usetikzlibrary{calc}

\usepackage{tabularx, booktabs}
\newcolumntype{Y}{>{\centering\arraybackslash}X}

\usepackage[inline]{enumitem}
\usepackage{refcount}%<-- non in WorksheetExam1

\usepackage{pstricks-add,pst-eucl}

\def\f{x+1} \def\g{-x/3+2}  \def\h{-x+3}

\newcommand{\vasymptote}[2][]{
	\draw [densely dashed,#1] ({rel axis cs:0,0} -| {axis cs:#2,0}) -- ({rel axis cs:0,1} -| {axis cs:#2,0});
}

\newcommand*\circled[1]{\tikz[baseline=(char.base)]{%
		\node[shape=circle,draw,inner sep=1pt] (char) {#1};}}
\renewcommand\choicelabel{\circled{\thechoice}}


\let\originalleft\left
\let\originalright\right
\renewcommand{\left}{\mathopen{}\mathclose\bgroup\originalleft}
\renewcommand{\right}{\aftergroup\egroup\originalright}

\addpoints
%\printanswers
\noprintanswers

\opengraphsfile{Q3a_Spring_16}

\begin{document}
\extrawidth{-0.3in}
\pagestyle{headandfoot}

\setlength{\hoffset}{-.25in}

\extraheadheight{-.3in}
\runningheadrule
\firstpageheader{\bfseries {MATH1-UC 1171}}{ \bfseries {Quiz 3 }}{\bfseries {2/16/2016}} 



\firstpagefooter{} %%&&CHANGED
                {}
                {Points earned: \hbox to 0.5in{\hrulefill}
                 out of  \pointsonpage{\thepage} points}
                 
						

\vspace*{0.1cm}
\hbox to \textwidth { \scshape {Name:} \enspace\hrulefill}
\vspace{0.1in}




\pointpoints{point}{points}

\begin{questions}


\addpoints

\question The graph of a function $f(x)$ is shown below.



\begin{center}

\begin{mfpic}[20]{-1}{6}{-2}{5}

%\polyline{(0,-2), (4,1), (4,2), (5,3)}

\polyline{(0,1), (3,1)} 

\polyline{(3,1), (4,2)}

\point[5pt]{(0,1), (4,2)}

\axes

\xmarks{1,2,3,4,5}

\ymarks{-2,-1,1,2,3,4,}

\tlpointsep{4pt}

\axislabels {x}{{\tiny $1$} 1, {\tiny $2$} 2, {\tiny $3$} 3, {\tiny $4$} 4, {\tiny $5$} 5}

\axislabels {y}{{\tiny $1$} 1,{\tiny $2$} 2, {\tiny $3$} 3, {\tiny $4$} 4,  {\tiny $-1$} -1, {\tiny $-2$} -2}

  % Grid
  %\drawcolor[gray]{0.25}
  %\gridlines{1, 1}
\drawcolor[gray]{0.75} 
\grid{1,1}

\end{mfpic}

\end{center}

\begin{parts}
\part[3] Fill out the table: 

\begin{minipage}{\linewidth}
\centering
  
%\begin{tabular}{|l|l|l|l|l|l|l|l|l|l|}
\begin{tabularx}{0.8\textwidth}{|X|X|X|X|X|X|X|X|X|X|X|}
\hline
\multicolumn{2}{|c|}{$x$}    & $-2$ & $-1$ & $0$ & $1$ & $2$ & $3$ & $4$ & $5$ & $6$ \\ \hline
\multicolumn{2}{|c|}{$f(x-1)$}  & &      &     &     &     &     &     &  &   \\ \hline
\end{tabularx}
\end{minipage}
\smallskip

\part[2] Write the domain of $f(x-1)$ in interval form. \dotfill 
\part[2] Write the range of $f(x-1)$ in interval form. \dotfill 
\part[2] Graph $\displaystyle f\left (\frac{x}{2}\right )$. [Hint: $\displaystyle f\left (\frac{x}{2}\right )$ is  a horizontal stretch of $f(x)$.] Label the tick marks on the $x$ and $y$ axes, and write the coordinates of the corner point.

\begin{center}

\begin{mfpic}[20]{-10}{10}{-2}{5}

%\polyline{(0,-2), (4,1), (4,2), (5,3)}

%\polyline{(0,1), (2,0)} 

%\polyline{(2,0), (4,2)}

%\point[5pt]{(0,1), (4,2)}

\axes

\xmarks{-10,-9,-8,-7,-6,-5,-4,-3,-2,-1,1,2,3,4,5,6,7,8,9,10}

\ymarks{-2,-1,1,2,3,4,}

\tlpointsep{4pt}

%\axislabels {x}{{\tiny $1$} 1, {\tiny $2$} 2, {\tiny $3$} 3, {\tiny $4$} 4, {\tiny $5$} 5}

%\axislabels {y}{{\tiny $1$} 1,{\tiny $2$} 2, {\tiny $3$} 3, {\tiny $4$} 4,  {\tiny $-1$} -1, {\tiny $-2$} -2}

  % Grid
  %\drawcolor[gray]{0.25}
  %\gridlines{1, 1}
\drawcolor[gray]{0.95} 
\grid{1,1}

\end{mfpic}

\end{center}


\end{parts}
\bonusquestion[2] Suppose the graph of a function $g$ is given. Describe how the graph of $\displaystyle y = 2g(x) - 1$ can be obtained from the graph of $g$ using function transformations.    
\fillwithdottedlines{3cm}
\bonusquestion[1] The function $\displaystyle h(x)=|x|$ is shifted 5 units to the right and shifted upward 6 units. Which of the following is the equation for the final transformed graph?
 
\begin{oneparchoices}
\choice $y=|x+5|+6$ \choice $y=|x+6|+5$	\choice $y=|x-5|+6$ \choice $y=|x-5|-6$ \choice $y=|x-6|+5$ \choice $y=|x-6|-5$ \choice $y=|x+5|-6$ \choice $y=|x+6|-5$
	\end{oneparchoices}
\end{questions}

\end{document}                 