\documentclass[11pt,dvipsnames]{article}
\usepackage[margin=0.4in,footskip=0.2in]{geometry}
\usepackage{etex}
\usepackage{amssymb,amsmath,multicol} %<-- InWorksheetExam1 i also have fancyhdr,

\usepackage[metapost]{mfpic}
\usepackage[pdftex]{graphicx}

\usepackage{pst-plot}
\usepackage{pgfplots}
\pgfplotsset{compat=1.9}

\usepackage{tikz}
\usepackage{tkz-2d}
\usepackage{tkz-base}
\usetikzlibrary{calc}
\usepackage{color}
\usepackage[inline]{enumitem}
\usepackage{refcount}%<-- non in WorksheetExam1

\usepackage[linewidth=1pt]{mdframed}

\usepackage{caption}
%\renewcommand{\headrulewidth}{0pt}

%%These three lines are for the typewriter font. Comment them out if I don't want the font.
%%%%%%\renewcommand*\ttdefault{lcmtt}
%%%%%%\renewcommand*\familydefault{\ttdefault} %% Only if the base font of the document is to be typewriter style
%%%%%%\usepackage[T1]{fontenc}
%%%%%%

\usepackage{tabularx, booktabs}


\newenvironment{myitemize}
{ \begin{itemize}
		\setlength{\itemsep}{10pt}
		\setlength{\parskip}{10pt}
		\setlength{\parsep}{10pt}     }
	{ \end{itemize}   
	
           } 


\opengraphsfile{Wk9_Questions_Spring_2016}

\begin{document}

%\includegraphics*[100,100][300,300]{nyu-scps-logo-lg.png}

\begin{enumerate}[label= {\bf  \arabic*:}]
\item In one of last week's discussion questions, we learned about the rule of 70. If $P$ dollars are invested at a rate of $r\%$ compounded yearly, then after $\displaystyle \frac{70}{r}$ years the balance of the account is close to $2P$ (so the initial amount has doubled.) We discovered that if $r=1\%$, then the rule of 70 gives a slight overestimate of the doubling time, while if $r=2\%$ it gives an underestimate, which is very close to the actual doubling time. If $r=5\%$ then the rule of 70 gives an underestimate again, but this time is less precise. Note that any time we have a quantity that grows at a constant rate of $r\%$ we can estimate the doubling time using the rule of 70. In this question, we will assume that the population of the Earth grows exponentially, at a annual growth rate of $r\%$, and we will look at the implications of this rapid growth.

\begin{enumerate}
	\item  The world population grows at a yearly rate of 1.13\% (so every year, the population is 1.13\% larger than the population the previous year.)
	\begin{enumerate}
	\item Use the rule of 70 to find the approximate doubling time of the population.
	\item Write an equation that you can use to find the exact value of the doubling time and solve it. (Hint: We solved this equation in class last Tuesday.) Then, use a calculator to find the exact doubling time.
	\item The Earth's population is 7.4 billion people. Assuming a population growth of 1.13\% per year, what will be the size of the population in 100 years?
	\end{enumerate}
\item  In this question we examine how a reduction in the yearly percent rate of growth affects the population size. Assume that the world population grows at a yearly rate of 1.05\% (so every year, the population is 1.05\% larger than the population the previous year.) 
	\begin{enumerate}
	\item Use the rule of 70 to find the approximate doubling time of the population.
	\item Write an equation that you can use to find the exact value of the doubling time and solve it. Then, use a calculator to find the exact doubling time.
	\item The Earth's population is 7.4 billion people. Assuming a population growth of 1.05\% per year, what will be the size of the population in 100 years?
	\end{enumerate}
\end{enumerate}
\item The population of Burkina Faso is about 18.9 million and grows at a rate of 3.03\% per year. The population of Canada is approximately 35 millions, and grows at a rate of 0.75\% per year. (The size of Burkina Faso is 105,869 square  miles, and the size of Canada is 3,854,085 square miles.) 
\begin{enumerate}
	\item Estimate the number of years it will take the population of Burkina Daso and the population of Canada to double (use the rule of 70.)
	\item Write a formula that gives the population of Burkina Faso $t$ years from now.
	\item Write a formula that gives the population of Canada $t$ years from now.
	\item Write an equation that allows you to find when the population of Burkina Faso and the population of Canada are equal.
\end{enumerate}	
\item The goal of this question is to compare the wealth of a millionaire and a billionaire graphically and numerically.
\begin{enumerate}
	\item \label{item:uno} Let's imagine a number line where each tick mark corresponds to the number of millions of dollars (so for example, the tick mark 1 stands for 1 million dollars, while the tick mark 5 stands for 5 million dollars.) 
	\begin{enumerate}
		\item At what number on the tape is the one billion dollar mark?
		\item If the marks on the tape are one centimeter apart, how many centimeters to the right of zero is the billion dollar mark? 
		\item How many meters to the right of zero is the billion dollar mark?
	\end{enumerate}
	\item \label{item:due} The billion dollar mark on the millionaire measuring tape in part \ref{item:uno} is way off the page. So let's use another number line, on which the mark at 1 represents one billion dollars.
	\begin{enumerate}
		\item Where is the 100 million dollar mark on this number line?
		\item Place marks at 1 million and 2 million dollars. Is it easy to visually distinguish the two marks?
		\item Place marks at one thousand and ten thousand dollars on the ruler.
	\end{enumerate}
	\item What makes the number lines in Questions \ref{item:uno} and \ref{item:due} not easy to use?
	\item Let's draw another number line. In this line, the number 1 represents $\displaystyle 10^1$, the number 2 represents $\displaystyle 10^2$ and so on. Note that each tick mark on this line represents a number that is ten times the number of the preceding mark.
	\begin{enumerate}
		\item Where is the one thousand dollar mark on this line? Where is the ten thousand dollar mark?
		\item Where are the one million and one billion dollar marks?
		\item Is there a one dollar mark on this line?
	\end{enumerate}
	\item Let's use another way to better understand the difference between a million dollars and a billion dollars.
	\begin{enumerate}
		\item \label{item:tre} Suppose you have one million dollars, and you decide to spend \$1000 per day. For how long would you be able to maintain this lifestyle before you become totally broke?
		\item Redo part \ref{item:tre} assuming that you have one billion dollars instead of one million.
	\end{enumerate}
\end{enumerate}
\end{enumerate}

\end{document}                 