\documentclass[11pt,dvipsnames]{article}

\usepackage{etex}
\usepackage{amssymb,amsmath,multicol} %<-- InWorksheetExam1 i also have fancyhdr,

\usepackage[metapost]{mfpic}
\usepackage[pdftex]{graphicx}

\usepackage{pst-plot}
\usepackage{pgfplots}
\pgfplotsset{compat=1.9}

\usepackage{tikz}
\usepackage{tkz-2d}
\usepackage{tkz-base}
\usetikzlibrary{calc}
\usepackage{color}
\usepackage[inline]{enumitem}
\usepackage{refcount}%<-- non in WorksheetExam1

\usepackage[linewidth=1pt]{mdframed}

\usepackage{caption}
%\renewcommand{\headrulewidth}{0pt}

%%These three lines are for the typewriter font. Comment them out if I don't want the font.
%%%%%%\renewcommand*\ttdefault{lcmtt}
%%%%%%\renewcommand*\familydefault{\ttdefault} %% Only if the base font of the document is to be typewriter style
%%%%%%\usepackage[T1]{fontenc}
%%%%%%

\newcommand{\vasymptote}[2][]{
	\draw [densely dashed,#1] ({rel axis cs:0,0} -| {axis cs:#2,0}) -- ({rel axis cs:0,1} -| {axis cs:#2,0});
}

\newcommand{\inlineitem}[1][]{%
	\ifnum\enit@type=\tw@
	{\descriptionlabel{#1}}
	\hspace{\labelsep}%
	\else
	\ifnum\enit@type=\z@
	\refstepcounter{\@listctr}\fi
	\quad\@itemlabel\hspace{\labelsep}%
	\fi}

\usetikzlibrary{automata}

\usepackage{hyperref}% http://ctan.org/pkg/hyperref

\usepackage{geometry}
\geometry{
	%a4paper,
	%total={170mm,257mm},
	%left=20mm,
	%top=20mm,
	text={.8\paperwidth,.9\paperheight}, ratio=1:1,includefoot
}



\opengraphsfile{Wk3_Questions_Spring_16}

\begin{document}
%\thispagestyle{empty}
						
\begin{enumerate}[label=$\blacktriangleright$ {\bf  \arabic*:}]
	\item The graph of a function $f(x)$ is shown below.



\begin{center}

\begin{mfpic}[20]{-1}{6}{-2}{5}

%\polyline{(0,-2), (4,1), (4,2), (5,3)}

\polyline{(0,1), (1,4)} 

\polyline{(1,4), (2,2)}

\polyline{(2,2), (3,4)}

\polyline{(3,4), (4,1)}

\point[5pt]{(0,1), (4,1)}
%\circle{(4, 2),0.15}

%\tlabel[cc](-1,1){\scriptsize $(0,1)$}

%\tlabel[cc](2,3.5){\scriptsize $(2,3)$}

%\tlabel[cc](4.5,2.5){\scriptsize $(4,2)$}

%\tlabel[cc](5,-0.5){\scriptsize $(5,0)$}

%\tlabel[cc](6,-0.5){\scriptsize $x$}

%\tlabel[cc](0.5,6){\scriptsize $y$}


\tcaption{\scriptsize $y=f(x)$}

\axes

\xmarks{1,2,3,4,5}

\ymarks{-2,-1,1,2,3,4,}

\tlpointsep{4pt}

\axislabels {x}{{\tiny $1$} 1, {\tiny $2$} 2, {\tiny $3$} 3, {\tiny $4$} 4, {\tiny $5$} 5}

\axislabels {y}{{\tiny $1$} 1,{\tiny $2$} 2, {\tiny $3$} 3, {\tiny $4$} 4,  {\tiny $-1$} -1, {\tiny $-2$} -2}

  % Grid
  %\drawcolor[gray]{0.25}
  %\gridlines{1, 1}
\drawcolor[gray]{0.75} 
\grid{1,1}

\end{mfpic}

\end{center}

	\begin{enumerate}[label=\textcolor{blue}{\bf (\alph*)}]

\item Fill out the table:


\begin{minipage}{\linewidth}
\centering
\captionof{table}{} \label{tab:title} 
\begin{tabular}{|l|l|l|l|l|l|l|l|}
\hline
$x$    & $-1$ & $0$ & $1$ & $2$ & $3$ & $4$ & $5$ \\ \hline
$f(x)$ &      &     &     &     &     &     &     \\ \hline
\end{tabular}
\end{minipage}



\item What are the domain and range of $f$? Write the formula for $f$ (hint: it is a piecewise defined function with four parts.)


\item \label{ettob} Fill out the table:

\begin{minipage}{\linewidth}
\centering
\captionof{table}{} \label{tab:ettobbotte}  
\begin{tabular}{|l|l|l|l|l|l|l|l|l|l|}
\hline
$x$    & $-1$ & $0$ & 0.5 & $1$ & 1.5 & $2$ & $3$ & $4$ & $5$ \\ \hline
$f(x+1)$ &      &     &     &     &     &     &   &  & \\ \hline
\end{tabular}
\end{minipage}

Note that if you start  with an $x$-value, for example $x=2$, the table asks you to pair this $x$ value with $f(2+1)$, which is 4 (refer to the graph of $f(x)$: in this graph, when $x=3$, then $f(3)=4$.)
\item \label{parapero} Plot the points from table \ref{tab:ettobbotte}. How does the graph of $f(x)$ compare with the new graph you have obtained from this table? What are the domain and range of your new graph?

\item Redo Questions \ref{ettob} and \ref{parapero}  using the table:

\begin{minipage}{\linewidth}
	\centering
	\captionof{table}{} \label{tab:ettob2}  
	\begin{tabular}{|l|l|l|l|l|l|l|l|l|l|}
		\hline
		$x$    & $0$ & $0.25$ & 0.5 & $0.75$ & 1 & $1.25$ & $1.5$ & $1.75$ & $2$ \\ \hline
		$f(2x)$ &      &     &     &     &     &     &   &  & \\ \hline
	\end{tabular}
\end{minipage}

\end{enumerate}
\newpage

\item The graph of a function $g(x)$ is shown below.






\pgfmathdeclarefunction{f}{1}{%
	\pgfmathparse{#1*#1*(-0.5)+2*#1+1}%
}
% or \pgfmathdeclarefunction{f}{1}{\pgfmathparse{#1*#1}}

\fbox{\begin{tikzpicture}[baseline=(current bounding box.north),scale=0.5]
	\begin{axis}[
	axis y line=center,
	axis x line=middle, 
	xmin=0,
	xmax=4,
	ymin=0,
	ymax=4,
	xlabel=\scalebox{1.5}{$x$},
	ylabel=$y$,
	x label style={at={(current axis.right of origin)},anchor=north, below=5mm},
	y label style={at={(current axis.above origin)},rotate=0,anchor=south east,left=5mm},
	clip=false,
	grid=both,
	minor xtick={0,1,...,4},
	minor ytick={1,2,...,4},
	enlarge x limits=0,
	scaled x ticks = true
	]
	\addplot[domain=0:4,blue,line width=2.0pt] {f(x)}; %no shift (clearly centered
	%\addplot {f(x-1)}; %little shift to the right
	%\addplot {f(2*x)}; %shifted nearly off the sheet
	\draw [draw=blue, fill=blue, thick] (axis cs: 0, 1) circle (5.0pt);
	\draw [draw=blue, fill=blue, thick] (axis cs: 4, 1) circle (5.0pt);
	\end{axis}
	\node[below, yshift=-4mm] at (current bounding box.south) {Function $g(x)$};
	\end{tikzpicture}
}     
\parbox[t]{11cm}{\vskip0pt
	Explain how the graph of $g$ must be changed in order to get each of the graphs shown below. (Sample answer: The graph in Fig. A is obtained by moving the graph of $g$ by 1 unit to the right.)
	
	
	
}

%%%%%%%%%%%%%%

\pgfplotsset{
	standard/.style={
		axis x line=middle,
		axis y line=middle,
		enlarge x limits=0.15,
		enlarge y limits=0.15,
		every axis x label/.style={at={(current axis.right of origin)},anchor=north west},
		every axis y label/.style={at={(current axis.above origin)},anchor=north east}
	}
}

\begin{figure}[h]
	\hspace*{\fill}
	\begin{tikzpicture}[scale=0.5]
	\begin{axis}[
	axis y line=center,
	axis x line=middle, 
	xmin=0,
	xmax=5,
	ymin=0,
	ymax=4,
	xlabel=\scalebox{1.5}{$x$},
	ylabel=\scalebox{1.5}{$y$},
	%x label style={at={(current axis.right of origin)},anchor=north, below=5mm},
	%y label style={at={(current axis.above origin)},rotate=0,anchor=south east,left=5mm},
	clip=false,
	grid=both,
	minor xtick={0,1,...,4},
	minor ytick={1,2,...,4},
	enlarge x limits=0,
	scaled x ticks = true
	]
	\addplot[domain=1:5,blue,line width=2.0pt] {f(x-1)}; %no shift (clearly centered
	%\addplot {f(x-1)}; %little shift to the right
	%\addplot {f(2*x)}; %shifted nearly off the sheet
	\draw [draw=blue, fill=blue, thick] (axis cs: 1, 1) circle (5.0pt);
	\draw [draw=blue, fill=blue, thick] (axis cs: 5, 1) circle (5.0pt);
	\end{axis}
	%\end{tikzpicture} 
	\node[below, yshift=-4mm] at (current bounding box.south) {Fig. A};
	\end{tikzpicture}\hspace*{\fill}
	\begin{tikzpicture}[scale=0.5]
	\begin{axis}[
	axis y line=center,
	axis x line=middle, 
	xmin=-1,
	xmax=4,
	ymin=0,
	ymax=4,
	xlabel=\scalebox{1.5}{$x$},
	ylabel=\scalebox{1.5}{$y$},
	%axis labels at tip,
	%x label style={at={(current axis.right of origin)},anchor=north, below=5mm},
	%y label style={at={(current axis.above origin)},rotate=0,anchor=north,left=5mm,yshift=1.5ex},
	clip=false,
	grid=both,
	minor xtick={0,1,...,4},
	minor ytick={1,2,...,4},
	enlarge x limits=0,
	scaled x ticks = true
	]
	\addplot[domain=-1:3,blue,line width=2.0pt] {f(x+1)}; %no shift (clearly centered
	%\addplot {f(x-1)}; %little shift to the right
	%\addplot {f(2*x)}; %shifted nearly off the sheet
	\draw [draw=blue, fill=blue, thick] (axis cs: -1, 1) circle (5.0pt);
	\draw [draw=blue, fill=blue, thick] (axis cs: 3, 1) circle (5.0pt);
	\end{axis}
	%\end{tikzpicture} 
	\node[below, yshift=-4mm] at (current bounding box.south) {Fig. B};
	\end{tikzpicture}\hspace*{\fill}
	\begin{tikzpicture}[scale=0.5]
	\begin{axis}[
	axis y line=center,
	axis x line=middle, 
	xmin=-4,
	xmax=0,
	ymin=0,
	ymax=4,
	xlabel=\scalebox{1.5}{$x$},
	ylabel=\scalebox{1.5}{$y$},
	%axis labels at tip,
	%x label style={at={(current axis.right of origin)},anchor=north, below=5mm},
	%y label style={at={(current axis.above origin)},rotate=0,anchor=north,left=5mm,yshift=1.5ex},
	clip=false,
	grid=both,
	minor xtick={-4,-3,...,-1},
	minor ytick={1,2,...,4},
	enlarge x limits=0,
	scaled x ticks = true
	]
	\addplot[domain=-4:0,blue,line width=2.0pt] {f(-x)}; %no shift (clearly centered
	%\addplot {f(x-1)}; %little shift to the right
	%\addplot {f(2*x)}; %shifted nearly off the sheet
	\draw [draw=blue, fill=blue, thick] (axis cs: -4, 1) circle (5.0pt);
	\draw [draw=blue, fill=blue, thick] (axis cs: 0, 1) circle (5.0pt);
	\end{axis}
	%\end{tikzpicture} 
	\node[below, yshift=-4mm] at (current bounding box.south) {Fig. C};
	\end{tikzpicture}\hspace*{\fill}
	%\caption{Two figures side by-side}
	%\label{fig:test}
\end{figure}
%%%%%%%%%%%%%%%%%%%%
\begin{figure}[h]
	\hspace*{\fill}
	\begin{tikzpicture}[scale=0.5]
	\begin{axis}[
	axis y line=center,
	axis x line=middle, 
	xmin=0,
	xmax=4,
	ymin=0,
	ymax=2,
	xlabel=\scalebox{1.5}{$x$},
	ylabel=\scalebox{1.5}{$y$},
	%x label style={at={(current axis.right of origin)},anchor=north, below=5mm},
	%y label style={at={(current axis.above origin)},rotate=0,anchor=south east,left=5mm},
	clip=false,
	grid=both,
	minor xtick={1,2,3,4},
	minor ytick={1,2},
	enlarge x limits=0,
	scaled x ticks = true
	]
	\addplot[domain=0:4,blue,line width=2.0pt] {0.5*f(x)}; %no shift (clearly centered
	%\addplot {f(x-1)}; %little shift to the right
	%\addplot {f(2*x)}; %shifted nearly off the sheet
	\draw [draw=blue, fill=blue, thick] (axis cs: 0, 0.5) circle (5.0pt);
	\draw [draw=blue, fill=blue, thick] (axis cs: 4, 0.5) circle (5.0pt);
	\end{axis}
	%\end{tikzpicture} 
	\node[below, yshift=-4mm] at (current bounding box.south) {Fig. D};
	\end{tikzpicture}\hspace*{\fill}
	\begin{tikzpicture}[scale=0.5]
	\begin{axis}[
	axis y line=center,
	axis x line=middle, 
	xmin=0,
	xmax=4,
	ymin=0,
	ymax=6,
	xlabel=\scalebox{1.5}{$x$},
	ylabel=\scalebox{1.5}{$y$},
	%axis labels at tip,
	%x label style={at={(current axis.right of origin)},anchor=north, below=5mm},
	%y label style={at={(current axis.above origin)},rotate=0,anchor=north,left=5mm,yshift=1.5ex},
	clip=false,
	grid=both,
	minor xtick={0,1,...,4},
	minor ytick={1,2,...,6},
	enlarge x limits=0,
	scaled x ticks = true
	]
	\addplot[domain=0:4,blue,line width=2.0pt] {2*f(x)}; %no shift (clearly centered
	%\addplot {f(x-1)}; %little shift to the right
	%\addplot {f(2*x)}; %shifted nearly off the sheet
	\draw [draw=blue, fill=blue, thick] (axis cs: 0, 2) circle (5.0pt);
	\draw [draw=blue, fill=blue, thick] (axis cs: 4, 2) circle (5.0pt);
	\end{axis}
	%\end{tikzpicture} 
	\node[below, yshift=-4mm] at (current bounding box.south) {Fig. E};
	\end{tikzpicture}\hspace*{\fill}
	\begin{tikzpicture}[scale=0.5]
	\begin{axis}[
	axis y line=center,
	axis x line=middle, 
	xmin=0,
	xmax=2,
	ymin=0,
	ymax=4,
	xlabel=\scalebox{1.5}{$x$},
	ylabel=\scalebox{1.5}{$y$},
	%axis labels at tip,
	%x label style={at={(current axis.right of origin)},anchor=north, below=5mm},
	%y label style={at={(current axis.above origin)},rotate=0,anchor=north,left=5mm,yshift=1.5ex},
	clip=false,
	grid=both,
	minor xtick={0,1,2},
	minor ytick={1,2,...,4},
	enlarge x limits=0,
	scaled x ticks = true
	]
	\addplot[domain=0:2,blue,line width=2.0pt] {f(2*x)}; %no shift (clearly centered
	%\addplot {f(x-1)}; %little shift to the right
	%\addplot {f(2*x)}; %shifted nearly off the sheet
	\draw [draw=blue, fill=blue, thick] (axis cs: 0, 1) circle (5.0pt);
	\draw [draw=blue, fill=blue, thick] (axis cs: 2, 1) circle (5.0pt);
	\end{axis}
	%\end{tikzpicture} 
	\node[below, yshift=-4mm] at (current bounding box.south) {Fig. F};
	\end{tikzpicture}\hspace*{\fill}
	%\caption{Two figures side by-side}
	%\label{fig:test}
\end{figure}
%%%%%%%%%%%%%%%%%%%%
\begin{figure}[h]
	\hspace*{\fill}
	\begin{tikzpicture}[scale=0.5]
	\begin{axis}[
	axis y line=center,
	axis x line=middle, 
	xmin=0,
	xmax=4,
	ymin=-3,
	ymax=1,
	xlabel=\scalebox{1.5}{$x$},
	ylabel=\scalebox{1.5}{$y$},
	%x label style={at={(current axis.right of origin)},anchor=north, below=5mm},
	%y label style={at={(current axis.above origin)},rotate=0,anchor=south east,left=5mm},
	clip=false,
	grid=both,
	minor xtick={1,2,3,4},
	minor ytick={-3,-2,-1,1},
	enlarge x limits=0,
	scaled x ticks = true
	]
	\addplot[domain=0:4,blue,line width=2.0pt] {-f(x)}; %no shift (clearly centered
	%\addplot {f(x-1)}; %little shift to the right
	%\addplot {f(2*x)}; %shifted nearly off the sheet
	\draw [draw=blue, fill=blue, thick] (axis cs: 0, -1) circle (5.0pt);
	\draw [draw=blue, fill=blue, thick] (axis cs: 4, -1) circle (5.0pt);
	\end{axis}
	%\end{tikzpicture} 
	\node[below, yshift=-4mm] at (current bounding box.south) {Fig. G};
	\end{tikzpicture}\hspace*{\fill}
	\begin{tikzpicture}[scale=0.5]
	\begin{axis}[
	axis y line=center,
	axis x line=middle, 
	xmin=0,
	xmax=8,
	ymin=0,
	ymax=3,
	xlabel=\scalebox{1.5}{$x$},
	ylabel=\scalebox{1.5}{$y$},
	%axis labels at tip,
	%x label style={at={(current axis.right of origin)},anchor=north, below=5mm},
	%y label style={at={(current axis.above origin)},rotate=0,anchor=north,left=5mm,yshift=1.5ex},
	clip=false,
	grid=both,
	minor xtick={0,2,...,8},
	minor ytick={1,2,3},
	enlarge x limits=0,
	scaled x ticks = true
	]
	\addplot[domain=0:8,blue,line width=2.0pt] {f(0.5*x)}; %no shift (clearly centered
	%\addplot {f(x-1)}; %little shift to the right
	%\addplot {f(2*x)}; %shifted nearly off the sheet
	\draw [draw=blue, fill=blue, thick] (axis cs: 0, 1) circle (5.0pt);
	\draw [draw=blue, fill=blue, thick] (axis cs: 8, 1) circle (5.0pt);
	\end{axis}
	%\end{tikzpicture} 
	\node[below, yshift=-4mm] at (current bounding box.south) {Fig. H};
	\end{tikzpicture}\hspace*{\fill}
	\begin{tikzpicture}[scale=0.5]
	\begin{axis}[
	axis y line=center,
	axis x line=middle, 
	xmin=0,
	xmax=4,
	ymin=0,
	ymax=4,
	xlabel=\scalebox{1.5}{$x$},
	ylabel=\scalebox{1.5}{$y$},
	%axis labels at tip,
	%x label style={at={(current axis.right of origin)},anchor=north, below=5mm},
	%y label style={at={(current axis.above origin)},rotate=0,anchor=north,left=5mm,yshift=1.5ex},
	clip=false,
	grid=both,
	minor xtick={0,1,2,3,4},
	minor ytick={1,2,...,4},
	enlarge x limits=0,
	scaled x ticks = true
	]
	\addplot[domain=0:4,blue,line width=2.0pt] {f(x)+1}; %no shift (clearly centered
	%\addplot {f(x-1)}; %little shift to the right
	%\addplot {f(2*x)}; %shifted nearly off the sheet
	\draw [draw=blue, fill=blue, thick] (axis cs: 0, 2) circle (5.0pt);
	\draw [draw=blue, fill=blue, thick] (axis cs: 4, 2) circle (5.0pt);
	\end{axis}
	%\end{tikzpicture} 
	\node[below, yshift=-4mm] at (current bounding box.south) {Fig. L};
	\end{tikzpicture}\hspace*{\fill}
	%\caption{Two figures side by-side}
	%\label{fig:test}
\end{figure}

The equation of $g(x)$ is: $\displaystyle g(x)=-0.5x^2+2x+1$. Can you write the equations of (some of) the graphs in Fig. A through L? 

\end{enumerate}
\end{document}                 