\documentclass[12pt,dvipsnames]{article}
\usepackage[margin=0.4in,footskip=0.1in]{geometry}
\usepackage{etex}
\usepackage{amssymb,amsmath,multicol} %<-- InWorksheetExam1 i also have fancyhdr,
\usepackage{hyperref}
\usepackage[metapost]{mfpic}
\usepackage[pdftex]{graphicx}
\usepackage{csquotes}
\usepackage{pst-plot}
\usepackage{pgfplots}
\pgfplotsset{compat=1.9}

\usepackage{tikz}
\usepackage{tkz-2d}
\usepackage{tkz-base}
\usetikzlibrary{calc}
\usepackage{color}
\usepackage[inline]{enumitem}
\usepackage{refcount}%<-- non in WorksheetExam1

\usepackage[linewidth=1pt]{mdframed}

\usepackage{caption}
\usetikzlibrary{calc,fit,intersections,shapes,calc}
\usetikzlibrary{backgrounds}



%%These three lines are for the typewriter font. Comment them out if I don't want the font.
%%%%%%\renewcommand*\ttdefault{lcmtt}
%%%%%%\renewcommand*\familydefault{\ttdefault} %% Only if the base font of the document is to be typewriter style
%%%%%%\usepackage[T1]{fontenc}
%%%%%%

\usepackage{tabularx, booktabs}


\newenvironment{myitemize}
{ \begin{itemize}
		\setlength{\itemsep}{10pt}
		\setlength{\parskip}{10pt}
		\setlength{\parsep}{10pt}     }
	{ \end{itemize}   
	
           } 

\usepackage{setspace}

\font\maxi=cminch scaled 100
\usepackage{tgadventor}
%\renewcommand*\familydefault{\sfdefault} %% Only if the base font of the document is to be sans serif
\usepackage[T1]{fontenc}
\newcommand*{\myfont}{\fontfamily{\sfdefault}\selectfont}
\usepackage{pacioli}
\usepackage[OT1]{fontenc}


\usepackage{AlegreyaSans} %% Option 'black' gives heavier bold face
%% The 'sfdefault' option to make the base font sans serif
%\renewcommand*\oldstylenums[1]{{\AlegreyaSansOsF #1}}

\newcommand*\circled[1]{\tikz[baseline=(char.base)]{%
		\node[shape=circle,fill=blue!20,draw,inner sep=2pt] (char) {#1};}}
\newcommand*\squared[1]{\tikz[baseline=(char.base)]{%
		\node[shape=rectangle,fill=green!20,draw,inner sep=2pt] (char) {#1};}}


\usepackage{array}
\newcolumntype{L}[1]{>{\raggedright\let\newline\\\arraybackslash\hspace{0pt}}m{#1}}
\newcolumntype{C}[1]{>{\centering\let\newline\\\arraybackslash\hspace{0pt}}m{#1}}
\newcolumntype{R}[1]{>{\raggedleft\let\newline\\\arraybackslash\hspace{0pt}}m{#1}}

\newcommand\T{\rule{0pt}{2.6ex}}       % Top strut
\newcommand\B{\rule[-1.2ex]{0pt}{0pt}} % Bottom strut


\usepackage{lastpage}
\usepackage{fancyhdr}
\pagestyle{fancy} 

\rfoot{{\small{Page \thepage\ of \pageref{LastPage}}}}
\cfoot{}
\renewcommand{\baselinestretch}{1.50}\normalsize

%\opengraphsfile{Wk14_Questions_Spring_16}

\begin{document}

\begin{itemize}
	\item[\protect\circled{17}] A yam has been taken out of an oven and left on a counter at room temperature. The temperature of the yam is $x$ (in degrees Fahrenheit) and the time it takes the yam to reach a temperature of $x$ degrees is $\displaystyle T(x)=745-157\ln\left ( x-60 \right)$ (in minutes). 
	\begin{enumerate}[label=\protect\squared{\arabic*}]
		\item What is the initial temperature of the yam, that is, the temperature right after it is taken out of the oven?
		\smallskip
		
		The initial temperature of the yam corresponds to $T(x)=0$.
		\begin{align*}
		0                       &= 745-157\ln\left ( x-60 \right)\\
		\frac{745}{157}         &= \ln\left ( x-60 \right)\\
		e^{\frac{745}{157}}     &= x-60 \\
		60+ e^{\frac{745}{157}} &= x\\
		\end{align*}
		\item What is the room temperature?
		\smallskip
		
		As time increases, the temperature of the yam drops towards room temperature, so as $T\to \infty$, $x\to $ room temperature.
		\begin{align*}
		T                                 &\to \infty \\
		745-157\ln\left ( x-60 \right )  &\to \infty \\
	   -157\ln\left ( x-60 \right )      &\to \infty \\
	    157\ln\left ( x-60 \right )      &\to -\infty \\
	    \ln\left ( x-60 \right )         &\to -\infty \\
	    x                                 &\to 60\\
		\end{align*}
		
		Note that we work backwards  through the transformations of $\ln (x)$ that result in $T(x)$.
		
		Another way to answer this question is to answer part \ref{item:poincare} first.
		\item List all the transformations, in order, that you need to apply to the graph of the function $\displaystyle f(x) =157\ln x$ in order to get the graph of the function $\displaystyle T(x)$.
		\begin{itemize}
			\item Horizontal shift by 60 units to the right.
			\item Reflection around the horizontal axis.
			\item Vertical shift 745 units up.
		\end{itemize}
		\item\label{item:poincare} Write a formula for the temperature $x$ of the yam as a function of the time $T$. (Your formula will be: $x=\ldots$)
		
		\begin{align*}
		T                            &= 745-157\ln\left ( x-60 \right ) \\
		T-745                        &= -157\ln\left ( x-60 \right )\\
		\frac{745-T}{157}            &= \ln\left ( x-60 \right )\\
		e^{\frac{745-T}{157}}        &= x-60 \\
		60+e^{\frac{745-T}{157}}     &= x \\
		\end{align*}
	\end{enumerate}

	\item[\protect\circled{19}] A quantity increases by 20\% every 4 hours. What is its doubling time?
	
	\smallskip
	
	We set $t$ be the time in hours, and $M_0$ be the amount at $t=0$. Since we have constant percent growth, the quantity is increasing exponentially. 
	
	\begin{minipage}{\linewidth}
		\centering
		%\captionof{table}{} %\label{tab:title} 
		\begin{tabular}{|L{1.5cm}|C{1.5cm}|C{1.5cm}|C{1.5cm}|C{1.5cm}|}
			\hline
			$t$      &  0     &   4     &    8          & 12   \T\B   \\ \hline
			$M(t)$   &  $M_0$ & $1.2M_0$& $1.2^2M_0$    &  $1.2^3M_0$           \T\B             \\ \hline
		\end{tabular}
	\end{minipage}

\begin{align*}
M(t)                  &=    M_02^{\frac{t}{h}}\\
M(4)                  &=    1.2M_0\\
M_02^{\frac{4}{h}}    &=    1.2M_0\\
2^{\frac{4}{h}}       &=    1.2\\
\frac{4}{h}           &=    \log_2(1.2)\\
\frac{4}{\log_2(1.2)} &=    h\\
\end{align*}


	\item[\protect\circled{25}] What are the transformations of $\displaystyle y=\sin\left(x\right)$ that result in $\displaystyle y=\sin\left (2 x-\frac{\pi}{3}\right)$? Does the order in which these transformations are performed matter? Why or why not?
	\smallskip
	
	A sequence of transformations is:
	\begin{enumerate}
		\item Horizontal shift by \protect\squared{{\scalebox{1.5}{$\frac{\pi}{\textcolor{red}{3}}$}}} to the right, followed by
		\item Horizontal compression by a factor of 2.
	\end{enumerate}
	Another sequence of transformations is:
	\begin{enumerate}
		\item Horizontal compression by a factor of 2, followed by
		\item Horizontal shift by \protect\squared{{\scalebox{1.5}{$\frac{\pi}{\textcolor{blue}{6}}$}}} to the right.
	\end{enumerate}
	
	\end{itemize}

\end{document} 
              