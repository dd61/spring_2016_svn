\documentclass[11pt,dvipsnames]{article}

\usepackage{etex}
\usepackage{amssymb,amsmath,multicol} %<-- InWorksheetExam1 i also have fancyhdr,

\usepackage[metapost]{mfpic}
\usepackage[pdftex]{graphicx}

\usepackage{pst-plot}
\usepackage{pgfplots}
\pgfplotsset{compat=1.9}

\usepackage{tikz}
\usepackage{tkz-2d}
\usepackage{tkz-base}
\usetikzlibrary{calc}
\usepackage{color}
\usepackage[inline]{enumitem}
\usepackage{refcount}%<-- non in WorksheetExam1
\usepackage{changepage}% http://ctan.org/pkg/changepage

%%%%\usepackage[linewidth=1pt]{mdframed}

%%%%\usepackage{caption}
%\renewcommand{\headrulewidth}{0pt}

%%These three lines are for the typewriter font. Comment them out if I don't want the font.
%%%%%%\renewcommand*\ttdefault{lcmtt}
%%%%%%\renewcommand*\familydefault{\ttdefault} %% Only if the base font of the document is to be typewriter style
%%%%%%\usepackage[T1]{fontenc}
%%%%%%

\usepackage{tabularx, booktabs}
\newcolumntype{Y}{>{\centering\arraybackslash}X}



%\usepackage{background}
%\newsavebox\mybox
%\savebox\mybox{\tikz[color=red!50,opacity=0.4]\node{Draft};}

\newcommand{\vasymptote}[2][]{
	\draw [densely dashed,#1] ({rel axis cs:0,0} -| {axis cs:#2,0}) -- ({rel axis cs:0,1} -| {axis cs:#2,0});
}

\newcommand{\inlineitem}[1][]{%
	\ifnum\enit@type=\tw@
	{\descriptionlabel{#1}}
	\hspace{\labelsep}%
	\else
	\ifnum\enit@type=\z@
	\refstepcounter{\@listctr}\fi
	\quad\@itemlabel\hspace{\labelsep}%
	\fi}

\usetikzlibrary{automata}

\usepackage{hyperref}% http://ctan.org/pkg/hyperref

\usepackage{geometry}
\geometry{
	%a4paper,
	%total={170mm,257mm},
	%left=20mm,
	%top=20mm,
	text={.8\paperwidth,.9\paperheight}, ratio=1:1,includefoot
}


\setlength{\parindent}{0pt} % disable paragraph indentation
\newlength{\linesepskip}
\setlength{\linesepskip}{2pt} % adjust to suit

\newcommand*{\linesep}[1]{%
	\par\nobreak
	\vskip 3pt \leaders\vrule width #1\vskip 0.81pt
	\nobreak
}

\usepackage{tabu}

%\backgroundsetup{angle=60,contents={\usebox\mybox}}
\begin{document}
	\begin{center}
		{\Large Classroom Observation Report}
	\end{center}
\begin{minipage}{0.35\textwidth}
		Dr.~Jason Samuels \par
		\linesep{\textwidth}
		Instructor
	\end{minipage}
\hspace{0.2\textwidth}
	\begin{minipage}{0.35\textwidth}
			SCNC1-UC2104 \par
			\linesep{\textwidth}
			Course
		\end{minipage}
		
\vskip 1cm
		
\begin{minipage}{0.35\textwidth}
	Donatella Delfino \par
	\linesep{\textwidth}
	Observer
\end{minipage}
\hspace{0.2\textwidth}
\begin{minipage}{0.35\textwidth}
	3/3/2016 \par
	\linesep{\textwidth}
	Date of observation
\end{minipage}

\vskip 1cm

\begin{minipage}{0.35\textwidth}
	11 \par
	\linesep{\textwidth}
	\# of students present in the week
\end{minipage}

\vskip 1cm

\begin{minipage}{0.35\textwidth}
	6 pm \par
	\linesep{\textwidth}
	Observer's arrival time
\end{minipage}
\hspace{0.2\textwidth}
\begin{minipage}{0.35\textwidth}
	7:30 pm \par
	\linesep{\textwidth}
	Observer's departure time
\end{minipage}

\vskip 1cm
{\large Observer}

Comment on and rate each item below and add comments in space provided:
\vskip 0.5cm

%%\begin{minipage}{\linewidth}
%%	\centering 
%%	\begin{tabular}[t]{|c|c|c|c|c|p{3cm}|}
%%		\hline
%%		5 = Excellent    & 4 =Very Good & 3 = Good & 2 = Fair & 1 = Poor & NA = Not appropriate for this class  \\ \hline
		
%%	\end{tabular}
%%\end{minipage}

{\tabulinesep=1.2mm
	\begin{tabu}{c  c c c c p{4cm} }
		
		5 = Excellent   & 4 =Very Good & 3 = Good & 2 = Fair & 1 = Poor & NA = Not appropriate for this class  \\ 	
	\end{tabu}}
\vskip 0.5cm

\begin{enumerate}[label= {\bf  \arabic*:}]
	\item
	\begin{tabular}[t]{p{0.6\textwidth} p{2cm} p{2cm} }
		Knowledge of subject & & Rating: 5 \\
			&&\\
			Students asked many questions on friction and conservation of momentum, and Dr.~Samuels answered with ease and confidence. & & \\ 
		%Dr. Kolokotronis is an evolutionary biologist with an established research program and a solid track record of publications. & & \\
	\end{tabular} 
	
	
	\item 		\begin{tabular}[t]{p{0.6\textwidth} p{2cm} p{2cm} }
		Clarity of learning objectives for the course as a whole and for the week's lesson within the course & & Rating: 5\\
	    & & \\
	    The objectives of the course (paraphrased from the syllabus) are to make students proficient in explaining the following concepts using the laws of physics: linear and rotational motion, bouncing and oscillations, gravitation and orbits, properties and actions of a gas, thermal energy, and sound. More broadly, the course aims to enable students to describe a variety of everyday phenomena (from riding a bicycle to bouncing a ball) using physics.
	    
	    The objectives of the class fit seamlessly into the course objectives: students reviewed friction, and learned about conservation of momentum in a closed system, represented by two bumper cars gliding on a smooth surface. && \\
	\end{tabular} 
	\item 		\begin{tabular}[t]{p{0.6\textwidth} p{2cm} p{2cm} }
		Clarity of criteria for assessing student performance for all activities & & Rating: 4 \\
		& & \\
		The assessment consists of online homework (on the online homework system WileyPlus), in-class quizzes, two exams and a paper. The syllabus provides a short description of the expectations for the paper, but does not outline requirements such as length, bibliography, etc. &  & \\
		\end{tabular} 
    \item 		\begin{tabular}[t]{p{0.6\textwidth} p{2cm} p{2cm} }
	Selection of appropriate instructional materials and methods & & Rating: 5\\
	& &\\
	The textbook, ``How Things Work'', by Louis Bloomfield, is very appropriate to introduce physics to non-majors. Indeed, one of the author's areas of expertise is the pedagogy of physics instruction for non STEM majors. The demonstration on bumper cars was a success: after Dr.~Samuels' brief introduction on conservation of momentum, most students demonstrated a basic understanding of the concept and were able to follow the demonstration, make predictions and ask pertinent questions. The demonstration led to a lively discussion of other situations where the conservation of momentum plays a crucial role (for example, a rocket launch.)
	
	When a student asks a question, Dr.~Samuels does not immediately volunteer the answer: instead, he pauses (the pauses can feel long, but they are actually just a few seconds) so that oftentimes either the student who has asked the question elaborates on it, making in clearer, or other students answer.  & &\\
\end{tabular} 
\item 		\begin{tabular}[t]{p{0.6\textwidth} p{2cm} p{2cm} }
	Effective time management including building of appropriate
	time frames for activities and assignments & & Rating: 5\\
	& &\\
	The pace of the course is appropriate and the activities within the lesson do not feel rushed or too slow. &&\\
	
\end{tabular}
\item 		\begin{tabular}[t]{p{0.6\textwidth} p{2cm} p{2cm} }
Evidence of faculty engagement, interaction with students, timely feedback and ongoing assessment & & Rating: 4\\
& &\\
The class is very interactive, because the questions that Dr.~Samuels asks do not have a simple yes/no answer: they are designed to make students think in terms of physics. Two of the questions were: ``As you pedal forward, what is exerting the force that the bicycle needs to accelerate? What would change if you were pedaling on ice?'' 

Students have to submit the topic of their paper to Dr.~Samuels, but there are no formal deadlines for the paper until the submission date of May 5th: students are not give the opportunity to submit a draft for critique, and may procrastinate working on the paper.
& &\\
\end{tabular}
\item 		\begin{tabular}[t]{p{0.6\textwidth} p{2cm} p{2cm} }
	Evidence of student engagement and interaction & & Rating: 4\\
	& &\\
	Most of the students were participating very actively: however, two students in the first row, and two students in the last row were not as vocal as the rest of the class. & &\\
\end{tabular}
\item 		\begin{tabular}[t]{p{0.6\textwidth} p{2cm} p{2cm} }
	Encouragement of critical and analytical thinking & & Rating: 5\\
\end{tabular}
%%\item 		\begin{tabular}[t]{p{0.6\textwidth} p{2cm} p{2cm} }
%%	Clarity of navigation & & Rating: 
%%\end{tabular}
%%\item 		\begin{tabular}[t]{p{0.6\textwidth} p{2cm} p{2cm} }
%%	Effective use of platform functionalities for organizing and
%%	pacing the material of the course and the work of the students & & Rating: 
%%\end{tabular}
\end{enumerate}
%\begin{enumerate}[label=$\hdots$ {\bf  \alph*:}]
\vskip 1cm
{\large Additional comments}

%\vskip 5cm %comment out

Here are some points that Dr.~Samuels and I discussed at our post observation meeting.
\begin{enumerate}[label=\textcolor{blue}{\bf (\alph*)}]
	\item Grading rubric for the paper.
	\begin{adjustwidth}{1cm}{1cm}
	I shared the grading rubric that I developed when I taught the course. The rubric is attached at the end of this document.
	\end{adjustwidth}
	\item What types of in-class demonstrations have you done so far? How did they go?
	\begin{adjustwidth}{1cm}{1cm}
	Demonstrations from the book as well as other demonstrations. For example, Dr.~Samuels plans to do the \emph{double ball bounce} demonstration (\url{http://phun.physics.virginia.edu/demos/double.html}) with two basketballs.
	\end{adjustwidth}
	\item Is there anything that you would like to change in the organization of the course? Why would you change it?
<<<<<<< .mine
	\begin{adjustwidth}{1cm}{1cm}
	It would be great to be able to go slightly faster, but it doesn't seem possible because the class meets only once a week.
	\end{adjustwidth}
	
=======
	\begin{adjustwidth}{1cm}{1cm}
	It would be great to be able to go slightly faster, but it doesn�t seem possible because
the class meets only once a week.
\end{adjustwidth}
%\item What is working well in the course? 
>>>>>>> .r216
\item What is the aspect of the course that you like the most?
<<<<<<< .mine
\begin{adjustwidth}{1cm}{1cm}
The class is fun to teach!
\end{adjustwidth}
%\item What is the aspect of the course that you like the most?
%\item What is the concept that students have found most difficult so far?
\item How is the online homework going?
\begin{adjustwidth}{1cm}{1cm}
The majority of the students are keeping up with the homework.
\end{adjustwidth} 

=======
\begin{adjustwidth}{1cm}{1cm}
The class is fun to teach!
\end{adjustwidth}
%\item What is the concept that students have found most difficult so far?
\item How is the online homework going? 
>>>>>>> .r216
\begin{adjustwidth}{1cm}{1cm}
The majority of the students are keeping up with the homework.
\end{adjustwidth}
	\end{enumerate}

\vskip 1cm
{\large Recomendations}
%\vskip 5cm %comment out

\begin{itemize}
	\item Create a timeline for the paper during the semester (for example, topic of the project, bibliography, draft of the paper).
	\item Consider asking questions to all students. 
	
	
	
\end{itemize}
\vskip 1cm
{\large Observer's evaluation of the class}

%CHANGE AS APPROPRIATE

{\tabulinesep=1.2mm
	\begin{tabu}{c  c c c c  }
		
		5 = Excellent   & {\fbox{4 =Very Good}} & 3 = Good & 2 = Fair & 1 = Poor   \\ 	
	\end{tabu}}

\end{document}                 