\documentclass[11pt,dvipsnames]{article}
\usepackage[margin=0.4in,footskip=0.2in]{geometry}
\usepackage{etex}
\usepackage{amssymb,amsmath,multicol} %<-- InWorksheetExam1 i also have fancyhdr,

\usepackage[metapost]{mfpic}
\usepackage[pdftex]{graphicx}

\usepackage{pst-plot}
\usepackage{pgfplots}
\pgfplotsset{compat=1.9}

\usepackage{tikz}
\usepackage{tkz-2d}
\usepackage{tkz-base}
\usetikzlibrary{calc}
\usepackage{color}
\usepackage[inline]{enumitem}
\usepackage{refcount}%<-- non in WorksheetExam1

\usepackage[linewidth=1pt]{mdframed}

\usepackage{caption}
%\renewcommand{\headrulewidth}{0pt}

%%These three lines are for the typewriter font. Comment them out if I don't want the font.
%%%%%%\renewcommand*\ttdefault{lcmtt}
%%%%%%\renewcommand*\familydefault{\ttdefault} %% Only if the base font of the document is to be typewriter style
%%%%%%\usepackage[T1]{fontenc}
%%%%%%

\usepackage{tabularx, booktabs}


\newenvironment{myitemize}
{ \begin{itemize}
		\setlength{\itemsep}{10pt}
		\setlength{\parskip}{10pt}
		\setlength{\parsep}{10pt}     }
	{ \end{itemize}   
	
           } 


\opengraphsfile{Wk9_Questions_Spring_2016}

\begin{document}
\begin{enumerate}[label= {\bf  \arabic*:}]
\item Suppose that we are investing \$100,000 at an interest rate of 1\%, compounded yearly. We leave the money and all interest in the account, and we don't withdraw any amount. We want to answer the question: after how many years will the amount in the account double? 
\begin{enumerate}
	\item  Fill out the table:


\begin{minipage}{\linewidth}
\centering

{\setlength{\tabcolsep}{1.9em}  
{\renewcommand{\arraystretch}{2}%
\begin{tabular}{|l|l|l|l|l|}
\hline
$t$    & $0$ & $30$ & $60$ & $90$  \\ \hline
$\displaystyle 100,000\left (1.01^t \right ) $ &      &     &   &\\ \hline
\end{tabular}}} \quad
\end{minipage}


\item  Find the amount in the bank after 70 years.


\item Assume that we are investing \$100,000 at an interest rate of 2\%, compounded yearly. Assuming that we don't make any additional deposits or withdrawals, how much money is in the account after 35 years?


\item  Assume that we are investing \$100,000 at an interest rate of 5\%, compounded yearly. Assuming that we don't make any additional deposits or withdrawals, how much money is in the account after 14 years?

 
\item \label{item:Pasqua} What is a reasonable guess for number of years it takes \$100,000 to double, if we have invested at an interest rate of $r$, compounded yearly? 

 
\item  Going back to question (\ref{item:Pasqua}), write an equation that allows us to find the exact number of years after which the investment has doubled. 
\end{enumerate}
\end{enumerate}

\noindent\fbox{%
	\parbox{\textwidth}{%
{\setlength{\parindent}{0cm} 
	A few points  that my question \emph{does not} make (but should make) are: When is the rule of 70 exact? When is it an underestimate? When is it an overestimate?
\smallskip

We note that if $r=1\%$ then the formula gives an overestimate. We also note that if $r=2$ the rule of 70 gives a slight underestimate (199988.75\$) and if $r=5$ it again gives an underestimate, which is worse than the one it gives if $r=2$. We guess that for an $r$ between 1 and 2, the rule of 70 gives an exact value. The doubling time for an investment earning an interest of $r$\%, compounded yearly, is given by $\displaystyle \frac{\ln 2}{\ln(1+r)}$. The question is: For what $r$ is 
$\displaystyle \frac{\ln 2}{\ln(1+r)}=\frac{70}{r}$?

We use the MacLaurin approximation of $\ln(1+r)$ (note that $|r|<1$ for convergence).
\begin{align*}
\ln(1+r)&\simeq r-\frac{r^2}{2}\\
\frac{\ln 2}{\ln(1+r)} &\simeq \frac{\ln 2}{r-\frac{r^2}{2}}\\
\frac{\ln 2}{r-\frac{r^2}{2}}&\simeq \frac{70}{r}\\
r\ln 2 &\simeq 70r-70\frac{r^2}{2}\\
\ln 2 &\simeq 70-70\frac{r}{2}\\
\ln 2 &\simeq 70-35r\\
35r &\simeq 70-\ln 2\\
r&\simeq 2-\frac{\ln 2}{35}\\
r&\simeq 1.9802\\
\end{align*}

}
}
}

\begin{enumerate}[label= {\bf  \arabic*:},resume]
\item Here are a few questions about the horizontal and vertical line test.
\begin{enumerate}
\item What is the vertical line test used for?
\item  What is the horizontal line test used for? 
\item Can you give an example of a function that passes the horizontal line test, but fails the vertical line test?
\item Can you give an example of a function that passes the vertical line test, but fails the horizontal line test?
\end{enumerate}
\item The function $\displaystyle f(x)=10^x$ has an inverse. (Why?) This inverse function is called: $\displaystyle f^{-1}(x)=\log_{10}x$, or we can omit the subscript 10 and call it simply $\displaystyle \log x$.
\begin{enumerate}
	\item What are the domain and range of $f(x)$?
	\item What are the domain and range of $\displaystyle f^{-1}(x)$?
	\item Since $f(x)$ has a horizontal asymptote, then $\displaystyle f^{-1}(x)$ has a vertical asymptote. What is the equation of this asymptote?
	\item  (Hint for this question: $\displaystyle \log x$ is the inverse of the function $\displaystyle 10^x$, so inputs of $\displaystyle \log x$ are outputs of $\displaystyle 10^x$.) Fill out the table:
	
	\begin{minipage}{\linewidth}
		\centering
		
		{\setlength{\tabcolsep}{1.3em}  
			{\renewcommand{\arraystretch}{2}%
				\begin{tabular}{|l|l|l|l|l|l|l|l|l|l|}
					\hline
					$x$    & $-1$ & $0$ & 0.1& $\displaystyle \sqrt{10}$ & $1$ & 10&100&1000  \\ \hline
				$\displaystyle \log x$ &      &     &   &&&&&\\ \hline
				\end{tabular}}} \quad
			\end{minipage}
	\item \label{item:Pasqua1} Fill out the table:
	
		\begin{minipage}{\linewidth}
			\centering
			
			{\setlength{\tabcolsep}{1.3em}  
				{\renewcommand{\arraystretch}{2}%
					\begin{tabular}{|l|l|l|l|l|l|l|l|l|l|}
						\hline
						$x$    & $-1$ & $0$ & 0.1& $\displaystyle \sqrt{10}$& $1$ & 10&100&1000  \\ \hline
						$\displaystyle 10^{\log x}$ &      &     &   &&&&&\\ \hline
					\end{tabular}}} \quad
				\end{minipage}	
	\item \label{item:Pasqua2} What general rule can you guess from the table in question (\ref{item:Pasqua1})?
	\item Why does the rule from question (\ref{item:Pasqua2})	work for all values of $x$?	
	\item \label{item:Pasqua23} Find $\displaystyle 10^{\log a + \log b}$ for the following values of $a$ and $b$:
	\medskip
	
	\begin{myitemize}
		\item[$\bigcirc$] $a=0.1, b=\sqrt{10}$ \hbox to 0.3\textwidth {\enspace\dotfill}
		%\item[$\bigcirc$] $a=0.1, b=10$ \hbox to 0.3\textwidth {\enspace\dotfill}
		\item[$\bigcirc$] $a=0.1, b=10$ \hbox to 0.3\textwidth {\enspace\dotfill}
		\item[$\bigcirc$] $a=0.1, b=100$ \hbox to 0.3\textwidth {\enspace\dotfill}
		\item[$\bigcirc$] $a=0.1, b=1000$ \hbox to 0.3\textwidth {\enspace\dotfill}
		\item[$\bigcirc$] $a=\sqrt{10}, b=10$ \hbox to 0.3\textwidth {\enspace\dotfill}
		\item[$\bigcirc$] $a=\sqrt{10}, b=100$ \hbox to 0.3\textwidth {\enspace\dotfill}
		\item[$\bigcirc$] $a=\sqrt{10}, b=1000$ \hbox to 0.3\textwidth {\enspace\dotfill}
		\item[$\bigcirc$] $a=10, b=10$ \hbox to 0.3\textwidth {\enspace\dotfill}
		\item[$\bigcirc$] $a=10, b=100$ \hbox to 0.3\textwidth {\enspace\dotfill}
		\item[$\bigcirc$] $a=10, b=1000$ \hbox to 0.3\textwidth {\enspace\dotfill}
	\end{myitemize}	
	\medskip
	
	\item What general rule can you guess from your answers to question (\ref{item:Pasqua23})?
		
\end{enumerate}

\end{enumerate}

\end{document}                 