\documentclass[12pt]{article}   % style for Physical Review B and AJP are similar

\usepackage{amsmath}    % need for subequations
\usepackage{hyperref}
\usepackage{color}
%\usepackage{fancybox}
%\usepackage{fancyhdr}
%\pagestyle{fancy}
\usepackage[top=3cm, bottom=3cm, left=3cm, right=3cm]{geometry}
%\rfoot{\thepage}
%\renewcommand{\headrulewidth}{0.4pt}
%\renewcommand{\footrulewidth}{0.4pt}

\begin{document}

\title{Y13.2104.001 Report on Object or Everyday Situation}  
\author{Donatella Delfino}
\date{\today}

\maketitle

{\sc{The Report}}

In class we have described how several objects, such as ramps and wind turbines, work. The goal of this report is for you to develop an analysis of the physics behind an object of your choice. I don't expect you to choose very complicated objects or situations, or objects with so many components that you can describe only a few of them: the objects may be simple, but you should devote time and thought to consider the details. Let's assume for a moment that you describe the way a skater moves on ice. You should note, for example, that friction and air resistance are involved (where??) You should also note that the way the skater holds himself (with open arms, with arms wrapped around his body) change his motion, and that other forces may be present (how does a skater jump?) You should consider these and other concepts to explain how the skater's movements are affected.

Your report must include sources. It needs to be original (no plagiarism), yours (no group work, or work with a tutor) and well written.
\smallskip

{\sc{Due Dates}}
\begin{enumerate}
\item A preliminary report is due by email on 3/2/2011. This report should include the object or situation that you plan to describe, and some of the physics concepts that are involved. 
\item The final report is due in class on the last day of class (5/11/2011).
\end{enumerate}
\smallskip

{\sc{Sources}}
If you use Internet sources, make sure that the sites contain correct information! I'd recommend that you use the sites listed in the syllabus as sources for your weekly article reports, and check with me about other Internet sources you are planning to include.
\smallskip

{\sc{Additional Help}}
I will be happy to discuss your progress in writing the report: please send me an email! Writing tutoring and recitation are offered to help you with your writing and organization of the material.
\smallskip

{\sc{Topics that are not permitted}}
The objects and situations  listed below have been discussed or will be discussed extensively in class, or are described in detail in your textbook. You may not focus your report on any of the objects and situations in this list.
\begin{enumerate}
\item A skater gliding on ice;
\item Falling balls;
\item Ramps;
\item A piano moving on a ramp;
\item Wind turbines;
\item Pinwheels;
\item Wheels;
\item Bumper cars;
\item Spring scales (including kitchen scales, grocery scales and bathroom scales);
\item Bouncing balls;
\item Carousels;
\item Roller coasters;
\item Bicycles (including tricycles);
\item Rockets;
\item Spacecrafts;
\item Balloons (including hot air balloons and birthday balloons);
\item Water pumps;
\item Balls used in sports (baseball, golf), beach balls;
\item Wood stoves.
\end{enumerate}

{\sc{Grading sheet}}
\smallskip

Note: This grading sheet borrows some of the contents of the rubric available at \url{http://rabi.phys.virginia.edu/1050/2003/tpgrade.html}.
\smallskip

Since the report counts for 15\% of the course grade, I will rescale the points you receive out of a maximum of 15 points.
\smallskip

{\textbf{Format [4 points]}}
\begin{itemize}
\item[$\bigcirc$] Your name is included.
\item[$\bigcirc$] The course number is included.
\item[$\bigcirc$] The paper is typed.
\item[$\bigcirc$] Page numbers are included.
\item[$\bigcirc$] The title is included.
\item[$\bigcirc$] The preliminary report (submitted on 3/2/2011) is included.
\item[$\bigcirc$] The paper is approximately 5 pages long (double spaced, 12 pt font).
\item[$\bigcirc$] The paper includes a bibliography.
\end{itemize}
\smallskip

{\textbf{Explanation of physics concepts and principles involved in the object [20 points]}}
\begin{itemize}
\item[$\bigcirc$] The paper lists and extensively describes the physics concepts and principles involved in the object [20 points].
\item[$\bigcirc$] The paper lists and briefly describes the physics concepts and principles involved in the object [15 points]. 
\item[$\bigcirc$] The paper lists  some of the physics concepts and principles involved in the object, but does not include descriptions [5 points].
\item[$\bigcirc$] The paper does not describe the physics concepts and principles involved in the object (for example, it focuses on how the design of the object has evolved over time) [0 points].

\end{itemize}
\smallskip

{\textbf{Writing (positives) [10 points]}}
\begin{itemize}
\item[$\bigcirc$] The writing is entertaining and engaging.

\item[$\bigcirc$] The paper is well organized, and it has a clear sense of beginning-middle-end.
\item[$\bigcirc$] The paper is free from grammatical errors.
\item[$\bigcirc$] Word choice is specific, purposeful, and varied throughout the paper.
\item[$\bigcirc$] Sentences are clear, active (Subject/Verb/Object), and to the point. 

\end{itemize}

{\textbf{Writing (Negatives)}}
\begin{itemize}
\item[$\bigcirc$] Grammatical errors distract from the reading.
\item[$\bigcirc$] The choice of words in the paper is often redundant and not specific. 
\item[$\bigcirc$] Sentences are somewhat unclear.
\item[$\bigcirc$] There is an excessive use of the passive voice.
\item[$\bigcirc$] The paper contains unsupported statements and/or opinions.
\item[$\bigcirc$] The paper is repetitive (for example, the same physics concept is described more than once but with different words).
\item[$\bigcirc$] The paper includes too much historical perspective (a couple of paragraphs is fine) [2 points deduction].
\item[$\bigcirc$] The paper includes a lengthy description of how the object is used (a couple of paragraphs is fine) [2 points deduction].
\item[$\bigcirc$] The paper is too long (more than a page over the 5-pages limit) [2 points deduction].
\item[$\bigcirc$] The paper is too short (more than one page under the 5-pages limit) [2 points deduction].
\end{itemize}

{\textbf{Plagiarism}}

The SCPS plagiarism policy will be strictly enforced. If you quote a sentence from a source, please use quotation marks and quote the source: failure to do so will result in a grade of F (zero points) for the paper. 

{\textcolor{red}{All papers will be checked for plagiarism using the website \url{http://www.turnitin.com}}}


%%%%%%%%%%%%%%%%%%%%%%%%%%%
%%%%%ERASE EVERYTHING FORM HERE DOWN.
%%%%%%%%%%%%%%%%%%%%%%%%%%%



\end{document}