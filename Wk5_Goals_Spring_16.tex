\documentclass[11pt,dvipsnames]{article}

\usepackage{etex}
\usepackage{amssymb,amsmath,multicol} %<-- InWorksheetExam1 i also have fancyhdr,

\usepackage[metapost]{mfpic}
\usepackage[pdftex]{graphicx}

\usepackage{pst-plot}
\usepackage{pgfplots}
\pgfplotsset{compat=1.9}

\usepackage{tikz}
\usepackage{tkz-2d}
\usepackage{tkz-base}
\usetikzlibrary{calc}
\usepackage{color}
\usepackage[inline]{enumitem}
\usepackage{refcount}%<-- non in WorksheetExam1

\usepackage[linewidth=1pt]{mdframed}

\usepackage{caption}
\usepackage{csquotes}

\usepackage{frcursive}
\usepackage[T1]{fontenc}
%\newcommand{\setfont}[2]{{\fontfamily{#1}\selectfont #2}}
\newenvironment{myfont}{\fontfamily{frc}\selectfont}{\par}
%\renewcommand{\headrulewidth}{0pt}

%%These three lines are for the typewriter font. Comment them out if I don't want the font.
%%%%%%\renewcommand*\ttdefault{lcmtt}
%%%%%%\renewcommand*\familydefault{\ttdefault} %% Only if the base font of the document is to be typewriter style
%%%%%%\usepackage[T1]{fontenc}
%%%%%%

\usepackage{tabularx, booktabs}
\newcolumntype{Y}{>{\centering\arraybackslash}X}

\newcommand{\vasymptote}[2][]{
	\draw [densely dashed,#1] ({rel axis cs:0,0} -| {axis cs:#2,0}) -- ({rel axis cs:0,1} -| {axis cs:#2,0});
}

\newcommand{\inlineitem}[1][]{%
	\ifnum\enit@type=\tw@
	{\descriptionlabel{#1}}
	\hspace{\labelsep}%
	\else
	\ifnum\enit@type=\z@
	\refstepcounter{\@listctr}\fi
	\quad\@itemlabel\hspace{\labelsep}%
	\fi}

\usetikzlibrary{automata}

\usepackage{hyperref}% http://ctan.org/pkg/hyperref

\usepackage{geometry}
\geometry{
	%a4paper,
	%total={170mm,257mm},
	%left=20mm,
	%top=20mm,
	text={.8\paperwidth,.9\paperheight}, ratio=1:1,includefoot
}


\opengraphsfile{Wk5_Overview_Spring_16}

\begin{document}
%\thispagestyle{empty}
\begin{center}
\begin{myfont}
{\Large Week 5 Learning Objectives }	
\end{myfont}
\end{center}


After completing this lesson, you should be able to:

\begin{enumerate}[label=\textcolor{blue}{\bf (\arabic*)}]

\item Graph parabolas using function transformations of $\displaystyle y = x^2$.
\item Identify the domain and range of a parabola, find all intercepts and the vertex.
\item Solve word problems involving polynomials of degree 2.
\item Identify the degree, the leading coefficient and the constant term of a polynomial function.
\item Describe the long-run behavior of a polynomial function.
\item Be familiar with the Intermediate Value Theorem for polynomial functions.
\item Describe the graph of a polynomial function near its x-intercepts.
\item Sketch qualitative graphs of polynomial functions.
\end{enumerate}
In order to actively participate in class, please review the discussion questions posted in the Lessons>Week 5 folder before class.


\end{document}                 