\documentclass[11pt,dvipsnames]{article}

\usepackage{etex}
\usepackage{amssymb,amsmath,multicol} %<-- InWorksheetExam1 i also have fancyhdr,

\usepackage[metapost]{mfpic}
\usepackage[pdftex]{graphicx}

\usepackage{pst-plot}
\usepackage{pgfplots}
\pgfplotsset{compat=1.9}

\usepackage{tikz}
\usepackage{tkz-2d}
\usepackage{tkz-base}
\usetikzlibrary{calc}
\usepackage{color}
\usepackage[inline]{enumitem}
\usepackage{refcount}%<-- non in WorksheetExam1

\usepackage[linewidth=1pt]{mdframed}

\usepackage{caption}
%\renewcommand{\headrulewidth}{0pt}

%%These three lines are for the typewriter font. Comment them out if I don't want the font.
%%%%%%\renewcommand*\ttdefault{lcmtt}
%%%%%%\renewcommand*\familydefault{\ttdefault} %% Only if the base font of the document is to be typewriter style
%%%%%%\usepackage[T1]{fontenc}
%%%%%%

\usepackage{tabularx, booktabs}
\newcolumntype{Y}{>{\centering\arraybackslash}X}

\newcommand{\vasymptote}[2][]{
	\draw [densely dashed,#1] ({rel axis cs:0,0} -| {axis cs:#2,0}) -- ({rel axis cs:0,1} -| {axis cs:#2,0});
}

\newcommand{\inlineitem}[1][]{%
	\ifnum\enit@type=\tw@
	{\descriptionlabel{#1}}
	\hspace{\labelsep}%
	\else
	\ifnum\enit@type=\z@
	\refstepcounter{\@listctr}\fi
	\quad\@itemlabel\hspace{\labelsep}%
	\fi}

\usetikzlibrary{automata}

\usepackage{hyperref}% http://ctan.org/pkg/hyperref

\usepackage{geometry}
\geometry{
	%a4paper,
	%total={170mm,257mm},
	%left=20mm,
	%top=20mm,
	text={.8\paperwidth,.9\paperheight}, ratio=1:1,includefoot
}


\opengraphsfile{Wk4_Questions_Spring_16}

\begin{document}
\thispagestyle{empty}
We can do function arithmetic ($+,-,\times, \div$) by defining the four operations in the obvious way. For example, if we start with two functions $f$ and $g$, the function $f+g$ is the rule that pairs each  $x$ to $f(x)+g(x)$.  {\bf{Important question}}: which $x$ values can I use to create $f+g, f-g,fg,\frac{g}{f}$? Let's try to answer it using an example.
						
The graph of two functions, $f(x)$ and $g(x)$, are shown below.


\begin{minipage}{0.5\linewidth}
\begin{center}

\begin{mfpic}[20]{-3}{6}{-2}{5}

\polyline{(-2,2), (2,0)} 
\polyline{(2,0), (4,2)}
\point[5pt]{(-2,2), (4,2)}
\tcaption{\scriptsize $y=f(x)$}
\axes
\xmarks{-2,-1,1,2,3,4,5}
\ymarks{-2,-1,1,2,3,4,}
\tlpointsep{4pt}
\axislabels {x}{{\tiny $-2$} -2,{\tiny $-1$} -1,{\tiny $1$} 1, {\tiny $2$} 2, {\tiny $3$} 3, {\tiny $4$} 4, {\tiny $5$} 5}
\axislabels {y}{{\tiny $1$} 1,{\tiny $2$} 2, {\tiny $3$} 3, {\tiny $4$} 4,  {\tiny $-1$} -1, {\tiny $-2$} -2}
\drawcolor[gray]{0.75} 
\grid{1,1}
\end{mfpic}
\end{center}
\end{minipage}
\begin{minipage}{0.5\linewidth}
\begin{center}

\begin{mfpic}[20]{-1}{6}{-2}{5}

%\polyline{(0,-2), (4,1), (4,2), (5,3)}

\polyline{(0,0), (3,2)} 

\polyline{(3,3), (5,4)}

\point[5pt]{(0,0), (3,2), (5,4)}
\circle{(3, 3),0.15}
\tcaption{\scriptsize $y=g(x)$}
\axes

\xmarks{1,2,3,4,5}

\ymarks{-2,-1,1,2,3,4,}

\tlpointsep{4pt}

\axislabels {x}{{\tiny $1$} 1, {\tiny $2$} 2, {\tiny $3$} 3, {\tiny $4$} 4, {\tiny $5$} 5}

\axislabels {y}{{\tiny $1$} 1,{\tiny $2$} 2, {\tiny $3$} 3, {\tiny $4$} 4,  {\tiny $-1$} -1, {\tiny $-2$} -2}

\drawcolor[gray]{0.75} 
\grid{1,1}

\end{mfpic}
\end{center}
\end{minipage}
\begin{enumerate}[label=$\blacktriangleright$ {\bf  \arabic*:}]

\item Fill out the table:


\begin{minipage}{\linewidth}
\centering
\captionof{table}{} %\label{tab:title} 
\begin{tabularx}{0.8\textwidth}{|X|X|X|X|X|X|X|X|X|X|}
\hline
\multicolumn{2}{|c|}{$x$}         &$-2$&$-1$& $0$ & $1$ & $2$ & $3$ & $4$ & $5$ \\ \hline
\multicolumn{2}{|c|}{$f(x)+g(x)$}   & & &     &     &     &     &     &     \\ \hline
\end{tabularx}
\end{minipage}

\item Find the domain and range of $f$, and the domain and range of $g$. Write them in interval form.
\item What is the domain of the function $f(x)+g(x)$?
\item Write formulas for $f(x), g(x)$ and $f(x)+g(x)$. 
\item Graph $f(x)+g(x)$.
\item \label{ettob} Fill out the table:

\begin{minipage}{\linewidth}
\centering
\captionof{table}{} \label{tab:ettob1}  
\begin{tabularx}{0.8\textwidth}{|X|X|X|X|X|X|X|X|X|X|}
\hline
\multicolumn{2}{|c|}{$x$}         & $-2$ & $-1$ & $0$ & $1$ & $2$ & $3$ & $4$ & $5$ \\ \hline
\multicolumn{2}{|c|}{$\displaystyle \frac{g(x)}{f(x)}$} & &      &     &     &     &     &     &     \\ \hline
\end{tabularx}
\end{minipage}

\item What is the domain of $\displaystyle \frac{g(x)}{f(x)}$?
\item Now we compose $f$ and $g$ by applying one after the other. For example, we can start with $g$, and feed all its outputs into $f$. The result is a new function, called $f(g(x))$. For example, $f(g(0))=f(0)=1$: we start with $x=0$ and look at the output given by $f$ when $x$ is zero. We see that this output is zero, so we feed it to the function $f$ and find $f(0)$.
\begin{enumerate}[label=\textcolor{blue}{\bf (\alph*)}]
	\item Fill out the table (note: you need to choose the $x$-values.)
	
	
	\begin{minipage}{\linewidth}
		\centering
		\captionof{table}{} \label{tab:ettob2}  
		\begin{tabularx}{0.8\textwidth}{|X|X|X|X|X|X|X|X|X|X|}
			\hline
			\multicolumn{2}{|c|}{$x$}         &  &  &  &  &  &  &  &  \\ \hline
			\multicolumn{2}{|c|}{$\displaystyle f(g(x))$} & &      &     &     &     &     &     &     \\ \hline
		\end{tabularx}
	\end{minipage}
\item What is the domain of $f(g(x))$?
\item Fill out the table (note: you need to choose the $x$-values.)


\begin{minipage}{\linewidth}
	\centering
	\captionof{table}{} \label{tab:ettob2}  
	\begin{tabularx}{0.8\textwidth}{|X|X|X|X|X|X|X|X|X|X|}
		\hline
		\multicolumn{2}{|c|}{$x$}         &  &  &  &  &  &  &  &  \\ \hline
		\multicolumn{2}{|c|}{$\displaystyle g(f(x))$} & &      &     &     &     &     &     &     \\ \hline
	\end{tabularx}
\end{minipage}
\item What is the domain of $g(f(x))$?		
	
	\item We can also compose a function with itself (it his famous 1976 paper on population growth, Prof.~Peter May composes a function with itself to get information about population trends). Fill out the table:
	
	
	\begin{minipage}{\linewidth}
		\centering
		\captionof{table}{} \label{tab:ettob3}  
		\begin{tabularx}{0.8\textwidth}{|X|X|X|X|X|X|X|X|X|X|}
			\hline
			\multicolumn{2}{|c|}{$x$}         &  &  &  &  &  &  &  &  \\ \hline
			\multicolumn{2}{|c|}{$\displaystyle g(g(x))$} & &      &     &     &     &     &     &     \\ \hline
		\end{tabularx}
	\end{minipage}
	\item What is the domain of $g(g(x))$?	
\end{enumerate}	
\item In the table shown below, is $x$ a function of $y$? (Note: I created this table by using the function values of $f(x)$.)


	\begin{minipage}{\linewidth}
		\centering
		\captionof{table}{} \label{tab:ettob4}  
		\begin{tabularx}{0.8\textwidth}{|X|X|X|X|X|X|X|X|X|}
			\hline
			\multicolumn{2}{|c|}{$x$}         & -2 & -1 & 0 & 1 & 2 & 3 & 4   \\ \hline
			\multicolumn{2}{|c|}{$\displaystyle y$} &2 &  1.5 &  1   &  0.5   &0     & 1    &  2      \\ \hline
		\end{tabularx}
	\end{minipage}
\end{enumerate}
\end{document}                 