\documentclass[12pt,dvipsnames]{article}
\usepackage[margin=0.4in,footskip=0.1in]{geometry}
\usepackage{etex}
\usepackage{amssymb,amsmath,multicol} %<-- InWorksheetExam1 i also have fancyhdr,
\usepackage{hyperref}
\usepackage[metapost]{mfpic}
\usepackage[pdftex]{graphicx}
\usepackage{csquotes}
\usepackage{pst-plot}
\usepackage{pgfplots}
\pgfplotsset{compat=1.9}

\usepackage{tikz}
\usepackage{tkz-2d}
\usepackage{tkz-base}
\usetikzlibrary{calc}
\usepackage{color}
\usepackage[inline]{enumitem}
\usepackage{refcount}%<-- non in WorksheetExam1

\usepackage[linewidth=1pt]{mdframed}

\usepackage{caption}
\usetikzlibrary{calc,fit,intersections,shapes,calc}
\usetikzlibrary{backgrounds}



%%These three lines are for the typewriter font. Comment them out if I don't want the font.
%%%%%%\renewcommand*\ttdefault{lcmtt}
%%%%%%\renewcommand*\familydefault{\ttdefault} %% Only if the base font of the document is to be typewriter style
%%%%%%\usepackage[T1]{fontenc}
%%%%%%

\usepackage{tabularx, booktabs}


\newenvironment{myitemize}
{ \begin{itemize}
		\setlength{\itemsep}{10pt}
		\setlength{\parskip}{10pt}
		\setlength{\parsep}{10pt}     }
	{ \end{itemize}   
	
           } 

\usepackage{setspace}

\font\maxi=cminch scaled 100
\usepackage{tgadventor}
%\renewcommand*\familydefault{\sfdefault} %% Only if the base font of the document is to be sans serif
\usepackage[T1]{fontenc}
\newcommand*{\myfont}{\fontfamily{\sfdefault}\selectfont}
\usepackage{pacioli}
\usepackage[OT1]{fontenc}


\usepackage{AlegreyaSans} %% Option 'black' gives heavier bold face
%% The 'sfdefault' option to make the base font sans serif
%\renewcommand*\oldstylenums[1]{{\AlegreyaSansOsF #1}}

\newcommand*\circled[1]{\tikz[baseline=(char.base)]{%
		\node[shape=circle,fill=blue!20,draw,inner sep=2pt] (char) {#1};}}

\usepackage{lastpage}
\usepackage{fancyhdr}
\pagestyle{fancy} 

\rfoot{{\small{Page \thepage\ of \pageref{LastPage}}}}
\cfoot{}
\renewcommand{\baselinestretch}{1.50}\normalsize

\opengraphsfile{Wk14_Questions_Spring_16}

\begin{document}
%\thispagestyle{empty}

	\begingroup
	\normalfont\AlegreyaSans
	\textcolor{blue}
	{
	By Tuesday (May 3) class you should have: (1) reviewed these questions and (2) prepared a sheet of formulas and examples that you may include in your note-card. You may not have the final version of your note-card yet, however, it is important that before the review on May 3 you have summarized the main points that the exam will cover. 
}
	\endgroup

\begin{center}
	%\section*{Pacioli}
	%\subsection*{\textbackslash cpcfamily}
	\begingroup
	\normalfont\cpcfamily
	\maxi {FUNDAMENTALS OF FUNCTIONS}
	\endgroup
\end{center}
\begin{enumerate}[label=\protect\circled{\arabic*}]
	\item What is a function? What are the domain and range? How do we test if a graph represents a function? How do we test if a function has an inverse?
	\item Can you graph a function that passes the vertical line test but not the horizontal line test?
		\item Can you graph a function that passes the horizontal line test but not the vertical line test?
		\item Is the unit circle a function?
		\item How are the various transformations (horizontal and vertical shifts, horizontal and vertical stretches, reflections about the axes) encoded in formulas is we start with an exponential function, a logarithmic function, or sine and cosine?
\end{enumerate}
\begin{center}
	%\section*{Pacioli}
	%\subsection*{\textbackslash cpcfamily}
	\begingroup
	\normalfont\cpcfamily
	\maxi {EXPONENTIAL AND LOGARITHMIC FUNCTIONS}
	\endgroup
\end{center}
	
\begin{enumerate}[label=\protect\circled{\arabic*},resume]
	\item What are the domain and range of $\displaystyle f(x)=a^x$? Can $a$ be any (fixed) number?
	\item How does the graph of an exponential function $\displaystyle f(x)=a^x$ (with $a>1$) differ from the graph of an exponential function $\displaystyle f(x)=b^x$ (with $0<b<1$)?
	\item What is the equation of the horizontal asymptote of $\displaystyle f(x)=a^x$?
	\item What do half life and doubling time mean?
	\item What are the domain and range of $\displaystyle g(x)=\log_a x$? 
	\item What are the domain and range of $\displaystyle g(x)=5\log_2 (3-4x)$? What is the equation of the vertical asymptote?
	\item Solve the equations:
	\begin{enumerate}
		\item $\displaystyle \log \left (x^2\right ) = 0$
		\item $\displaystyle \log x+ \log (x+1) = 1$
		\item $\displaystyle \log (x^2-1) = 10$
		\item $\displaystyle \log (x+3) -\log(1-x)=\log(6)$
		\item $\displaystyle 10^{x^2-1} = 2$
	\end{enumerate}

\end{enumerate}		
	


\begin{center}
	%\section*{Pacioli}
	%\subsection*{\textbackslash cpcfamily}
	\begingroup
	\normalfont\cpcfamily
	\maxi {COMPOUND INTEREST AND APPLICATIONS}
	\endgroup
	
\end{center}

\begin{enumerate}[label=\protect\circled{\arabic*},resume]
	\item How much do I need to invest now, at a yearly interest rate of 1.5\% compounded quarterly, if I want to have \$200,000 in my account 5 years from now?
\item The formula for the amount in an account after $t$ years (if the yearly interest rate is $r\%$ and the compounding happens $n$ times a year) is $\displaystyle P(t)=P_0\left (1+\frac{r}{n}\right ) ^{nt}$ (you should know this formula or copy it in your note-cards. Note that in the formula $r$ is written in decimal form, so for example if the yearly interest rate is 1\%, you will use 0.01 for $r$.) If $P_0,r,t$ don't change, and $n$ becomes larger and larger, does $P(t)$ grow without bound, or does it get closer and closer to a number?
\item I invest \$1000 at a yearly interest rate of 1\% compounded monthly. I also invest \$1200 at a yearly interest rate of 0.1\% compounded yearly. When will the two investments be equal? (Set up an equation and find the exact value for the number of years.)
\item I invest \$1000 at a yearly interest rate of 1\% compounded monthly. In how many years will the amount in the account be \$1100?
	\item A yam has been taken out of an oven and left on a counter at room temperature. The temperature of the yam is $x$ (in degrees Fahrenheit) and the time it takes the yam to reach a temperature of $x$ degrees is $\displaystyle T(x)=745-157\ln\left ( x-60 \right)$ (in minutes). 
	\begin{enumerate}
		\item What is the initial temperature of the yam, that is, the temperature right after it is taken out of the oven?
		\item What is the room temperature?
		\item List all the transformations, in order, that you need to apply to the graph of the function $\displaystyle f(x) =157\ln x$ in order to get the graph of the function $\displaystyle T(x)$.
		\item Write a formula for the temperature $x$ of the yam as a function of the time $T$. (Your formula will be: $x=\ldots$)
	\end{enumerate}
	\item A quantity decreases by 20\% every 4 hours. What is its half life?
	\item A quantity increases by 20\% every 4 hours. What is its doubling time?
\end{enumerate}

\begin{center}
	%\section*{Pacioli}
	%\subsection*{\textbackslash cpcfamily}
	\begingroup
	\normalfont\cpcfamily
	\maxi {TRIGONOMETRY}
	\endgroup
	
\end{center}

\begin{enumerate}[label=\protect\circled{\arabic*},resume]
	\item Why don't $\displaystyle y=\sin(x)$ and $\displaystyle y=\cos(x)$ have an inverse?
	\item What are the domain and range of $\displaystyle y=\sin^{-1}(x)$ and $\displaystyle y=\cos^{-1}(x)$?
	\item Which part(s) of the unit circle should you use to find values of  $\displaystyle\sin^{-1}(x)$ and $\displaystyle \cos^{-1}(x)$?
	\item Why is this statement false? $\displaystyle \frac{7\pi}{4}=\sin^{-1}\left (-\frac{\sqrt{2}}{2}\right )$
	\item What are the similarities and differences of the graphs of $\displaystyle y=\sin\left (2x\right)$ and $\displaystyle y=\sin\left (2\pi x\right)$?
	\item What are the transformations of $\displaystyle y=\sin\left(x\right)$ that result in $\displaystyle y=\sin\left (2 x-\frac{\pi}{3}\right)$? Does the order in which these transformations are performed matter? Why or why not?
	\end{enumerate}

\end{document} 
              