\documentclass[10pt,stdletter]{newlfm}
\usepackage{charter}

\widowpenalty=1000
\clubpenalty=1000

\newsavebox{\Luiuc}
\sbox{\Luiuc}{%
    \parbox[b]{1.75in}{%
        \vspace{0.5in}%
        \includegraphics[scale=1.0,ext=.png]
        {nyu-scps-logo-lg.png}%
    }%
}%

\makeletterhead{Uiuc}{\Lheader{\usebox{\Luiuc}}}

\newlfmP{headermarginskip=20pt}
\newlfmP{sigsize=50pt}
\newlfmP{dateskipafter=20pt}
%\newlfmP{addrfromphone}
\newlfmP{addrfromemail}
%\PhrPhone{Phone}
\PhrEmail{Email}

\lthUiuc

\namefrom{Donatella Delfino}
\addrfrom{%
   Donatella Delfino, Ph.D.\\
   Clinical Associate Professor\\
    NYU SPS Paul McGhee Division\\
    7 E 12th St, New York NY 10003
}

%\phonefrom{000-000-0000}
\emailfrom{dd61@nyu.edu}

\addrto{%
Search Committee Chairperson\\
Department of Computer Science\\
Hofstra University}

\greetto{To Whom It May Concern,}
\closeline{Sincerely,}
\begin{document}
\begin{newlfm}

I am pleased to support the candidacy of Dr.~Christos Noutsos for consideration for the tenure-track faculty position in computer science with emphasis in big data.

In September 2015, I started coordinating the scientific issues courses at the Paul McGhee Division at New York University (I have been coordinating the mathematics course at McGhee since 2003). Dr.~Noutsos was a new adjunct faculty, scheduled to teach a human genetics course.  Students in the course are non-majors, and do not necessarily have an extensive background in mathematics and science. The challenge of the human genetics course is to present the rich and complex course material in a way that is accessible and engaging to non-majors, while at the same time retaining academic rigor and integrity. Dr.~Noutsos brilliantly rose to the challenge.

Our initial conversations focused on pedagogy. Dr.~Noutsos had decided to require the students to create an annotated bibliography and research journals: the main challenge was to create a scaffolding to ease the students into online consultation of databases, formulation of a project question, selection of sources and annotations. Dr.~Noutsos created a semester calendar where students would submit parts of their work for review and grading. He was very open to suggestions; indeed, he requested them and engaged in a conversation about students' expectations and course objectives.

Since every new McGhee faculty is observed in the first and third semester of teaching,  I conducted a formal course observation of Dr.~Noutsos' class last October. Dr.~Noutsos was in command of the material, and he had visibly established a rapport with his students. He had posted a PowerPoint presentation on the NYU course management system, and students were eagerly following and asking questions. As an observer, I could not interact with the students, however, I knew a few of them from previous classes. Afterwards, one of the students approached me and shared that he had been intimidated by the title and course description of the class, but  was pleasantly surprised at the amount of material that he had learned and understood. 

This spring, Dr.~Noutsos is teaching the human genetics course again, and he has enriched the course by including two laboratory sessions on bioinformatics. Additionally, Dr~Noutsos will teach a new 2-credit course which will be a hands-on introduction to bioinformatics for non-majors. 

While Dr.~Noutsos is of course a strong researcher,  his abilities are not only academic: he is a team player who is generous with his time and expertise. He is open to criticism,  cares deeply about his students' learning, and he is responsive and thorough. These traits are the best indication of a professor in the most expansive and truest sense: they are the reasons that I strongly recommend Dr.~Noutsos  for the tenure-track position in computer science at Hofstra.

\end{newlfm}

\end{document}