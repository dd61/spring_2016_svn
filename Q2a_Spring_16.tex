\documentclass[11pt,answers]{exam}


\usepackage[margin=0.8in,footskip=0.2in]{geometry}

\usepackage{etex}
\usepackage{amssymb,amsmath,multicol} %<-- InWorksheetExam1 i also have fancyhdr,

\usepackage[metapost]{mfpic}
\usepackage[pdftex]{graphicx}

\usepackage{pst-plot}
\usepackage{pgfplots}
\pgfplotsset{compat=1.9}

\usepackage{tikz}
\usepackage{tkz-2d}
\usepackage{tkz-base}
\usetikzlibrary{calc}

\usepackage[inline]{enumitem}
\usepackage{refcount}%<-- non in WorksheetExam1

\usepackage{pstricks-add,pst-eucl}

\def\f{x+1} \def\g{-x/3+2}  \def\h{-x+3}

\newcommand{\vasymptote}[2][]{
    \draw [densely dashed,#1] ({rel axis cs:0,0} -| {axis cs:#2,0}) -- ({rel axis cs:0,1} -| {axis cs:#2,0});
}

\addpoints
%\printanswers
\noprintanswers

\opengraphsfile{Q2a_Spring_16}

\begin{document}
%%%%%\extrawidth{-0.4in}
\pagestyle{headandfoot}

\setlength{\hoffset}{-.25in}

%%%%%\extraheadheight{-.3in}
\runningheadrule
\firstpageheader{\bfseries {MATH1-UC 1171}}{ \bfseries {Quiz 2 }}{\bfseries {2/9/16}} 



\firstpagefooter{} %%&&CHANGED
                {}
                {Points earned: \hbox to 0.5in{\hrulefill}
                 out of  \pointsonpage{\thepage} points}
                 
						

\vspace*{0.1cm}
\hbox to \textwidth { \scshape {Name:} \enspace\hrulefill}
\vspace{0.1in}




\pointpoints{point}{points}

\begin{questions}


\addpoints

\bonusquestion[1]  A school fund-raising group sells chocolate bars to help finance a swimming pool for their physical education program. 
The group finds that when they set their price at $x$ dollars per bar (where 
$0 < x \leq 5$), their total sales revenue (in dollars) is given by the function 
$\displaystyle R(x) = −500x^2 + 3000x$. What do these values represent? (Circle the best answer.)

\begin{choices}
	\choice $R(2)$ and $R(4)$ represent their total expenses when their price is \$2 and \$4 per bar respectively.
	
	\choice $R(2)$ and $R(4)$ represent their total profits when their price is \$2 and \$4 per bar respectively.
	

	\choice $R(2)$ and $R(4)$ represent their total sales revenue when their price is \$2 and \$4 per bar respectively.
	
	\choice $R(2)$ and $R(4)$ represent their total profits when their profit is \$2 and \$4 per bar respectively.
	
	\choice $R(2)$ and $R(4)$ represent their total expenses when their expense is \$2 and \$4 per bar respectively.
\end{choices}





\question Yesterday I started charging my phone at 9:02 pm. The percent charge of my phone $t$ minutes after 9:02 pm is represented by the function $\displaystyle f(t)=1.17t+5$.

\begin{parts}
\part[1] $f(0)=$\dotfill
\part[1] What does $f(0)$ represent in practical terms? \dotfill
\part[1] Write an equation that will allow you to find the time when the phone is fully charged. Do not solve the equation and do not find the time! 
\fillwithdottedlines{1cm}	
\end{parts}

\question The complete graph of a function $f(x)$ is given below.

\begin{center}
	
	\begin{mfpic}[20]{-1}{6}{-2}{5}
		
		%\polyline{(0,-2), (4,1), (4,2), (5,3)}
		
		\polyline{(0,1), (5,1)} 
		
		%\polyline{(4,2), (5,2)}
		
		\point[5pt]{(0,1),  (5,2)}
		\circle{(5, 1),0.15}
		
		%\tlabel[cc](-1,1){\scriptsize $(0,1)$}
		
		%\tlabel[cc](2,3.5){\scriptsize $(2,3)$}
		
		%\tlabel[cc](4.5,2.5){\scriptsize $(4,2)$}
		
		%\tlabel[cc](5,-0.5){\scriptsize $(5,0)$}
		
		%\tlabel[cc](6,-0.5){\scriptsize $x$}
		
		%\tlabel[cc](0.5,6){\scriptsize $y$}
		
		
		%\tcaption{\scriptsize $y=f(x)$}
		
		\axes
		
		\xmarks{1,2,3,4,5}
		
		\ymarks{-2,-1,1,2,3,4,}
		
		\tlpointsep{4pt}
		
		\axislabels {x}{{\tiny $1$} 1, {\tiny $2$} 2, {\tiny $3$} 3, {\tiny $4$} 4, {\tiny $5$} 5}
		
		\axislabels {y}{{\tiny $1$} 1,{\tiny $2$} 2, {\tiny $3$} 3, {\tiny $4$} 4,  {\tiny $-1$} -1, {\tiny $-2$} -2}
		
		% Grid
		%\drawcolor[gray]{0.25}
		%\gridlines{1, 1}
		\drawcolor[gray]{0.75} 
		\grid{1,1}
		
	\end{mfpic}
	
\end{center}

\begin{parts}
	\part[1] $f(4.5)=$\dotfill
	\part[2] The domain (in interval form) is \dotfill
	\part[2] The range (in interval form) is \dotfill
	\part[2] Write a formula for $f(x)$.
	\fillwithdottedlines{2cm}

	
	
	%\makebox[0.6\linewidth]{{\bf{Range:}}\enspace\dotfill}
	
\end{parts}






%%%%%%%%%%%%%%%%%%%%%%%%%%



\end{questions}

\end{document}                 