\documentclass[11pt,dvipsnames]{article}
\usepackage[margin=0.4in,footskip=0.2in]{geometry}
\usepackage{etex}
\usepackage{amssymb,amsmath,multicol} %<-- InWorksheetExam1 i also have fancyhdr,
\usepackage{hyperref}
\usepackage[metapost]{mfpic}
\usepackage[pdftex]{graphicx}

\usepackage{pst-plot}
\usepackage{pgfplots}
\pgfplotsset{compat=1.9}

\usepackage{tikz}
\usepackage{tkz-2d}
\usepackage{tkz-base}
\usetikzlibrary{calc}
\usepackage{color}
\usepackage[inline]{enumitem}
\usepackage{refcount}%<-- non in WorksheetExam1

\usepackage[linewidth=1pt]{mdframed}

\usepackage{caption}
\usetikzlibrary{calc,fit,intersections,shapes,calc}
\usetikzlibrary{backgrounds}



%%These three lines are for the typewriter font. Comment them out if I don't want the font.
%%%%%%\renewcommand*\ttdefault{lcmtt}
%%%%%%\renewcommand*\familydefault{\ttdefault} %% Only if the base font of the document is to be typewriter style
%%%%%%\usepackage[T1]{fontenc}
%%%%%%

\usepackage{tabularx, booktabs}


\newenvironment{myitemize}
{ \begin{itemize}
		\setlength{\itemsep}{10pt}
		\setlength{\parskip}{10pt}
		\setlength{\parsep}{10pt}     }
	{ \end{itemize}   
	
           } 

\usepackage{setspace}

\renewcommand{\baselinestretch}{1.50}\normalsize

\opengraphsfile{Wk12_Questions_Spring_2016}

\begin{document}
\thispagestyle{empty}
%\includegraphics*[100,100][300,300]{nyu-scps-logo-lg.png}
The hours of daylight over the years and  the blood pressure over time (\url{http://cnx.org/contents/A4QcTJ6a@3/Blood-Flow-Blood-Pressure-and-} are two of the most obvious examples of phenomena that repeat periodically over time. The functions we have studied so far (linear, polynomial, rational, exponential and logarithmic) do not lend themselves well to model this type of phenomena. This week and next week we will work with two trigonometric functions (the sine function and the cosine function) that can be used to model some types of repetitive phenomena. Let's start by visualizing a few examples.  
\begin{enumerate}[label= {\bf  \arabic*:}]
\item A Ferris wheel with 60 capsules has a diameter of 110 meters and its lowest point (at the 6 o'clock position) is 5 meters above the ground. One full rotation of the wheel takes 24 minutes, and the wheel rotates counterclockwise at a constant speed. The capsules are labeled from 1 to 60. We start observing when capsule 1 is at the lowest point on the wheel. 
\begin{enumerate}
\item Let $H(t)$ be the height of capsule 1 $t$ minutes after we start the observation. Fill out this table:

\begin{minipage}{\linewidth}
\centering
%\captionof{table}{} %\label{tab:title} 
\begin{tabularx}{0.8\textwidth}{|X|X|X|X|X|X|X|X|X|X|X|X|X|}
\hline
\multicolumn{2}{|c|}{$t$}         &0&6& 12 & 18 & 24 & 2 & 3 & 4 &8&9&10\\ \hline
\multicolumn{2}{|c|}{$H(t)$}   & & &     &     &     &     &     && & &     \\ \hline
\end{tabularx}
\end{minipage}

\item Draw the graph of $H(t)$ over two rotations of the wheel.
\item The function $H(t)$ is periodic because it repeats itself every 24 minutes. More generally, if a function $f(t)$ completes one full cycle in a time interval of length $c$, then $c$ is called the period of this function (note that the period is the smallest time interval in which the function completes one cycle). The midline of a periodic function is the horizontal line that lies halfway between the peaks and the valleys of the graph. The amplitude of a periodic function is half of the vertical distance between the peaks and the valleys. What are the midline and the amplitude of the function $H(t)$?
\item During the first rotation, for how many minutes is capsule 1 at a height of at least 65 meters from the ground?
\item How would the graph change if we assume that the lowest point of the wheel is 4 meters above the ground?
\item How would the graph change if we assume that we start observing when capsule 1 is at the 3 o'clock position? 
\item How would the graph change if we assume that the wheel rotates clockwise?

\end{enumerate}
 \item A table of value for the periodic function $f(t)$ is shown below. What is the period of the function?
 
 \begin{minipage}{\linewidth}
\centering
%\captionof{table}{} %\label{tab:title} 
\begin{tabularx}{0.8\textwidth}{|X|X|X|X|X|X|X|X|X|X|X|X|X|}
\hline
\multicolumn{2}{|c|}{$t$}         &0&1& 2 & 3 & 4 & 5 & 6 & 7 &8&9&10\\ \hline
\multicolumn{2}{|c|}{$f(t)$}   & 12&13 & 14    &12     &13     &14     &12     &13&14 &12 & 13    \\ \hline
\end{tabularx}
\end{minipage}

\item A table of value for the periodic function $f(t)$ is shown below. What is the period of the function?
 
 \begin{minipage}{\linewidth}
\centering
%\captionof{table}{} %\label{tab:title} 
\begin{tabularx}{0.8\textwidth}{|X|X|X|X|X|X|X|X|X|X|X|}
\hline
\multicolumn{2}{|c|}{$t$}         &1&11& 21 & 31 & 41 & 51 & 61 & 71 &81\\ \hline
\multicolumn{2}{|c|}{$g(t)$}   & 5&3 & 2    &3     &5     &3     &2     &3&5 \\ \hline
\end{tabularx}
\end{minipage}

\end{enumerate}

\end{document}                 