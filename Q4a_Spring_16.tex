\documentclass[11pt,answers]{exam}


\usepackage[margin=0.8in,footskip=0.2in]{geometry}

\usepackage{etex}
\usepackage{amssymb,amsmath,multicol} %<-- InWorksheetExam1 i also have fancyhdr,

\usepackage[metapost]{mfpic}
\usepackage[pdftex]{graphicx}

\usepackage{pst-plot}
\usepackage{pgfplots}
\pgfplotsset{compat=1.9}

\usepackage{tikz}
\usepackage{tkz-2d}
\usepackage{tkz-base}
\usetikzlibrary{calc}

\usepackage{tabularx, booktabs}
\newcolumntype{Y}{>{\centering\arraybackslash}X}

\usepackage[inline]{enumitem}
\usepackage{refcount}%<-- non in WorksheetExam1

\usepackage{pstricks-add,pst-eucl}
\usepackage{caption}

\def\f{x+1} \def\g{-x/3+2}  \def\h{-x+3}

\newcommand{\vasymptote}[2][]{
	\draw [densely dashed,#1] ({rel axis cs:0,0} -| {axis cs:#2,0}) -- ({rel axis cs:0,1} -| {axis cs:#2,0});
}

\newcommand*\circled[1]{\tikz[baseline=(char.base)]{%
		\node[shape=circle,draw,inner sep=1pt] (char) {#1};}}
\renewcommand\choicelabel{\circled{\thechoice}}


\let\originalleft\left
\let\originalright\right
\renewcommand{\left}{\mathopen{}\mathclose\bgroup\originalleft}
\renewcommand{\right}{\aftergroup\egroup\originalright}

\addpoints
%\printanswers
\noprintanswers

\usepackage{geometry}
\geometry{
	%a4paper,
	%total={170mm,257mm},
	%left=20mm,
	%top=20mm,
	text={.95\paperwidth,.95\paperheight}, ratio=1:3,%includefoot
}

\opengraphsfile{Q4a_Spring_16}

\begin{document}
%\extrawidth{-0.3in}
\pagestyle{headandfoot}

%\setlength{\hoffset}{-.25in}

%\extraheadheight{-.3in}
\runningheadrule
\firstpageheader{\bfseries {MATH1-UC 1171}}{ \bfseries {Quiz 4 }}{\bfseries {2/23/2016}} 



\firstpagefooter{} %%&&CHANGED
                {}
                {Points earned: \hbox to 0.5in{\hrulefill}
                 out of  \pointsonpage{\thepage} points}
                 
						

%\vspace*{0.1cm}
\hbox to \textwidth { \scshape {Name:} \enspace\hrulefill}
\vspace{0.1in}




\pointpoints{point}{points}

\begin{questions}


\addpoints


\question A tank holds 100 gallons of water, which drains from a leak at the bottom, causing the tank to drain in 40 minutes. The volume of water remaining in the tank after $t$ minutes is $\displaystyle V(t)=100\left (1-\frac{t}{40}\right )^2$.
\begin{parts}
	\part[2]  Find $\displaystyle V^{-1}(100)$. Show your work step by step.
	\fillwithdottedlines{2cm} 
	\part[1] Does $\displaystyle V^{-1}(100)$ represent time or volume? \dotfill
	\end{parts}  

\question[2] The function shown below is not one-to-one.

\begin{minipage}{0.5\linewidth}
\begin{mfpic}[15]{-3}{3}{0}{5.5}
	\arrow \reverse \arrow \function{-2.2,2.2,0.1}{x**2}
	%\point[3pt]{(-2,4), (-1,1), (0,0), (1,1), (2,4)}
	\axes

	\xmarks{-2,-1,1,2}
	\ymarks{1,2,3,4}
	\tcaption{$f(x) = x^2$}
	\tlpointsep{4pt}
	\axislabels {x}{{\tiny $-2 \hspace{7pt}$} -2, {\tiny $-1 \hspace{7pt}$} -1, {\tiny $1$} 1, {\tiny $2$} 2}
	\axislabels {y}{{\tiny $1$} 1, {\tiny $2$} 2, {\tiny $3$} 3, {\tiny $4$} 4}
\end{mfpic}
\end{minipage}
\begin{minipage}{0.5\linewidth}
	 Restrict its domain so that the resulting function is one-to-one. Write your answer in interval form. (Note: there are many possible correct answers).
	 \fillwithdottedlines{2cm}
	 \end{minipage}
\question[3] A function $h(x)$ is shown below.

\begin{minipage}{0.5\linewidth}
	\begin{center}
		
		\begin{mfpic}[20]{-1}{3}{-1}{5}
			
			%\polyline{(0,-2), (4,1), (4,2), (5,3)}
			
			\polyline{(0,3), (2,3)} 
		
			
			\point[5pt]{(0,3), (2,3)}
			
			%\tcaption{\scriptsize $y=h(x)$}
			\axes
			
			\xmarks{1,2,3}
			
			\ymarks{-1,1,2,3,4,}
			
			\tlpointsep{4pt}
			
			\axislabels {x}{{\tiny $1$} 1, {\tiny $2$} 2, {\tiny $3$} 3}
			
			\axislabels {y}{{\tiny $1$} 1,{\tiny $2$} 2, {\tiny $3$} 3, {\tiny $4$} 4}
			
			\drawcolor[gray]{0.75} 
			\grid{1,1}
			
		\end{mfpic}
	\end{center}
\end{minipage}
\begin{minipage}{0.5\linewidth}
	Fill out the table. If a table entry does not exist, write DNE.
	\smallskip
	
	\centering
	%\captionof{table}{} %\label{tab:title} 
	\begin{tabularx}{0.8\textwidth}{|X|X|X|X|X|}
		\hline
		\multicolumn{2}{|c|}{$x$}         &$0$&$1$& $2$  \\ \hline
		\multicolumn{2}{|c|}{$h(h(x))$}   &   &   &      \\ \hline
	\end{tabularx}
\end{minipage}
\question Consider the following functions:
$\displaystyle f(x) = x-2$,     $\displaystyle g(x) = \sqrt{x}$ 
\begin{parts}
	\bonuspart[1] Find $\displaystyle f(4)+g(4)$. \dotfill
	\bonuspart[1] Find $\displaystyle \frac{g(4)}{f(4)}$ \dotfill
	\part[1] Write the domain of $\displaystyle f(x)+g(x)$ in interval form. \dotfill
	\part[1] Write the domain of $\displaystyle \frac{g(x)}{f(x)}$ in interval form. \dotfill
	\end{parts}
\end{questions}

\end{document}                 